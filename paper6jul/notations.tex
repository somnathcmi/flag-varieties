
   \section{Background and notations}
   \label{sec:notations}

We first recall some well known results about the Grassmannian.  The proofs of these statements can be found in \cite{lakshmibai2007standard} and \cite{seshadriintroduction}. Let $G=SL(n)$ and consider the
natural action of $G$ in ${\mathbb C}^n$. We fix a standard basis of ${\mathbb C}^n$, $e_1,\ldots,e_n$. We write the elements of $G$ with respect to this basis. 

Let $T$ be the maximal torus of diagonal matrices in $G$ and $B$ be the Borel subgroup of upper triangular matrices. $B^{-}$ is the group of lower triangular matrices and $T = B \cap B^{-}$. The Weyl group of $G$, $W$, is the group of permutations of $[n]$ and we have the Bruhat decomposition $G = \bigcup_w BwB$, where $w$ runs over all permutations of $W$.

Let $I_{r,n}:=\{{\bf \underline{i}} =(i_1,i_2,\ldots,i_r) |  1\leq i_1 < i_2 \cdots < i_r \leq n \}$ be the set of all strictly increasing sequences of legth $r$ with entries in $[n]$. 
 A canonical basis of $\bigwedge^r{\mathbb C}^n$ is given by $\{e_{\bf \underline{i}} = e_{i_1} \wedge \ldots \wedge e_{i_r}, {\bf \underline{i}} \in I_{r,n}\}$. 
 We view the Grassmannian as a subvariety of ${\mathbb P}(\bigwedge^r {\mathbb C}^n)$,
given by sending an $r$-dimensional
subspace of ${\mathbb C}^n$ spanned by vectors $v_1,v_2,\ldots,v_r$ to the class $[v_1 \wedge v_2 \wedge \cdots v_r] \in {\mathbb P}(\wedge^r {\mathbb C}^n)$.  The Grassmannian can be identified with the orbit of $G/P_{r}$, where $P_{r}$ is the maximal parabolic subgroup of $G$ fixing the subspace spanned by $[e_1,e_2, \ldots,e_r]$. The Weyl group of $P_{r}$, $W_{P_r}$, is a subgroup of $W$. The coset representatives of $W/W_{P_r}$  of minimal length, $W^{P_r}$, can be identified with $I_{r,n}$, with the coset representative $w$ corresponding to the subspace spanned by $[e_{w(1)},\ldots,e_{w(r)}]$. Via this identification the point
$e_w = e_{w(1)} \wedge e_{w(2)} \cdots \wedge e_{w(r)}$ in the Grassmannian is a $T$ fixed point of $G_{r,n}$. The $B$ orbit closure of $e_{w}$ is called a Schubert variety in $\grass{r}{n}$ and we denote it
by $X(w)$.  \eat{$X(w)$ is the union of the $B$-orbits of $e_{w'}$, $w' \leq w$.} 

Denote by $\{p_{\bf \underline{i}} | {\bf \underline{i}} \in I_{r,n}\}$ the dual basis of the canonical basis in  $(\bigwedge^r{\mathbb C}^n)^*$. The $p_{\bf \underline{i}}$ are called Pl\"{u}cker coordinates on the
 Grassmannian. The homogeneous coordinate ring of the Grassmannian ${\mathbb C}[\grass{r}{n}]$ is the quotient of ${\mathbb C}[p_{\bf \underline{i}} | {\bf \underline{i}} \in I_{r,n}]$ by the ideal of polynomials 
 $I(G_{r,n})$ vanishing on the Grassmannian for its embedding in ${\mathbb P}(\bigwedge^r{\mathbb C}^n)$.  
 
 We define a partial order "$\leq$" on $I_{r,n}$ by defining $ {\bf \underline{i}} \leq {\bf \underline{j}} \Leftrightarrow i_t \leq j_t$ for all $t = 1,2,\cdots,r$. The set $I_{r,n}$ with this partial order is called the Br\"{u}hat poset.  It will be useful to refine the partial order to a total order which we will also denote by the same symbol. Then there is a natural lexicographic order on ${\mathbb C}[\bigwedge^r {\mathbb C}^n]$ which is a 
 monomial order.
 
 A monomial $p_{{\bf \underline{i}}^1} p_{{\bf \underline{i}}^2} \cdots p_{{\bf \underline{i}}^s}$ of degree $s$ in the Pl\"{c}ker coordinates is said to be standard in ${\mathbb C}[\bigwedge^r {\mathbb C}^n]$ iff 
 ${\bf \underline{i}}^1 \leq {\bf \underline{i}}^2 \cdots \leq {\bf \underline{i}}^s$.  If a product of monomials $p_{\bf \underline{i}} p_{\bf \underline{j}}$ is not standard then it is known that 
 \[ p_{\bf \underline{i}} p_{\bf \underline{j}} = p_{\bf \underline{i} \cup \bf \underline{j}}p_{\bf \underline{i} \cap \bf \underline{j}} + \text{other quadratic terms\ \ } \text{modulo\ \ }  I(G_{r,n}) \]
 Here $({\bf \underline{i} \cup \bf \underline{j}})_t = max(i_t, j_t), t = 1,\ldots,r$ and  $({\bf \underline{i} \cap \bf \underline{j}})_t = min (i_t, j_t), t = 1,\ldots,r$. 
 Furthermore the other quadratic terms in the above expression are products of monomials which are strictly bigger than $p_{\bf \underline{i}} p_{\bf \underline{j}}$ in the monomial order on ${\mathbb C}[\bigwedge^r {\mathbb C}^n]$. Quadratic terms in the above expression which are nonstandard can be refined again using the same rule and, since the terms are increasing in lex order, we see that any non standard monomial has an expression in terms of standard monomials. The difference between the two expressions goes by the name "straightening law" and this is an element of the ideal $I(G_{r,n})$. As a result we have:
 \begin{theorem} 
 \begin{itemize}
 \item[i] Standard monomials span the coordinate ring of the Grassmannian.
 \item[ii]  The ideal $I(G_{r,n})$ is generated by straightening laws.
  \end{itemize}
\end{theorem}

If $w \in W^{P_r}$ corresponds to the element ${\bf \underline{i}}  \in I_{r,n}$, the Schubert variety $X(w) \subset G_{r,n}$ is given by the vanishing of the $p_{\bf \underline{j}}, {\bf \underline{j}} \not \leq  {\bf \underline{i}}$.

 \subsection{$T$-action on Pl\"{u}cker coordinates}
Since we are interested in the $T$ quotient of the Grassmannian, it is natural to consider the action of $T$ on the coordinate ring ${\mathbb C}[\grass{r}{n}]$.  A diagonal matrix $t$ with entries 
$(t_1,t_2,\ldots,t_n)$ acts on $p_{\bf \underline{i}}$, ${\bf \underline{i}} = (i_1,\ldots,i_r)$, by scaling it by $(t_{i_1}t_{i_2}\ldots t_{i_r})^{-1}$. It is clear that the action of $t$ on a monomial in
the Pl\"{u}cker coordinates $p_{{\bf \underline{i}}^1} p_{{\bf \underline{i}}^2} \cdots p_{{\bf \underline{i}}^s}$ is $(t_1^{\# 1} t_2^{\# 2}\ldots t_n^{\# n})^{-1}$ where $\# t$ is the number of $k$ such that
$t$ occurs in ${\bf \underline{i}}^k$. It is clear that the monomial is invariant under $T$ if for each $t \in [n]$, the number of $k$ for which $t$ occurs in ${\bf \underline{i}}^k$ is the same. 
Now assume that $(r,n)=1$. It is clear from the discussion so far that if $p_{{\bf \underline{i}}^1} p_{{\bf \underline{i}}^2} \cdots p_{{\bf \underline{i}}^s}$ is invariant,
$s$ must be a multiple of $n$. It is easy to see that there are monomials of degree $n$ in the Pl\"{u}cker coordinates which are $T$-invariant, see \cite{bakshi2020torus} for example. It follows:

\begin{theorem}
Let $R(m) = H^0(\grass{r}{n}, {\cal L}(nm))^T$. Then $\gmnmodt{r}{n}{\cal L}(\omega_r) = Proj(\oplus_m R(m))$.
\end{theorem}

Since $H^0(\grass{r}{n}, {\cal L}(nm))$ is spanned by standard monomials, the homogeneous coordinate ring of the $T$-quotient, $R=\oplus_m R(m)$, is spanned by $T$-invariants which are standard. We call these standard $T$-invariants.  Now standardness is defined by the "$\leq$" relation on the Br\"{u}hat poset. So a set of $T$-invariant monomials which are products of Pl\"{u}cker coordinates with support in a maximal chain in the Bruhat poset is naturally standard. Taking the union of such $T$-invariant monomials over all maximal chains in the Bruhat poset gives us a generating set for the ring $R$. This follows since every standard monomial which is $T$-invariant Pl\"{u}cker coordinates with are pairwise comparable  and the Bruhat poset has a minimal element and a maximal element. This is the approach we take.

We first formulate the question of obtaining $T$-invariant monomials with support in a fixed maximal chain of the Bruhat poset.
