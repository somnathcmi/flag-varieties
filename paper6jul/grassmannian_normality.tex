\section{The case of $Gr_{3,7}$}
\label{sec:g37}
In this section we give a computational proof that $(\Lgmnmodt{3}{7}{\cal L}(7\omega_3),{\cal M})$ is projectively normal. Before proceeding we illustrate the notation from the previous section.

\subsubsection{Example}

For \(\grass{3}{7}\), the Br\"{u}hat poset is on the set \(I(3,7)\) which has cardinality 35. There are 462 maximal chains in the Br\"{u}hat poset with lowest element \([1,2,3]\) and top element \([5,6,7]\). The length of every maximal chain is \(13\). Let \(\SC_0\) be the following maximal chain in \(I_{3,7}\). 

\[
\SC_0 = 
\begin{array}[c]{*{1}c}\cline{1-1}
\lr{1}\\\cline{1-1}
\lr{2}\\\cline{1-1}
\lr{3}\\\cline{1-1}
\end{array}
\le
\begin{array}[c]{*{1}c}\cline{1-1}
\lr{1}\\\cline{1-1}
\lr{2}\\\cline{1-1}
\lr{4}\\\cline{1-1}
\end{array}
\le
\begin{array}[c]{*{1}c}\cline{1-1}
\lr{1}\\\cline{1-1}
\lr{2}\\\cline{1-1}
\lr{5}\\\cline{1-1}
\end{array}
\le
\begin{array}[c]{*{1}c}\cline{1-1}
\lr{1}\\\cline{1-1}
\lr{2}\\\cline{1-1}
\lr{6}\\\cline{1-1}
\end{array}
\le
\begin{array}[c]{*{1}c}\cline{1-1}
\lr{1}\\\cline{1-1}
\lr{3}\\\cline{1-1}
\lr{6}\\\cline{1-1}
\end{array}
\le
\begin{array}[c]{*{1}c}\cline{1-1}
\lr{2}\\\cline{1-1}
\lr{3}\\\cline{1-1}
\lr{6}\\\cline{1-1}
\end{array}
\le
\begin{array}[c]{*{1}c}\cline{1-1}
\lr{2}\\\cline{1-1}
\lr{3}\\\cline{1-1}
\lr{7}\\\cline{1-1}
\end{array}
\le
\begin{array}[c]{*{1}c}\cline{1-1}
\lr{2}\\\cline{1-1}
\lr{4}\\\cline{1-1}
\lr{7}\\\cline{1-1}
\end{array}
\le
\begin{array}[c]{*{1}c}\cline{1-1}
\lr{3}\\\cline{1-1}
\lr{4}\\\cline{1-1}
\lr{7}\\\cline{1-1}
\end{array}
\le
\begin{array}[c]{*{1}c}\cline{1-1}
\lr{3}\\\cline{1-1}
\lr{5}\\\cline{1-1}
\lr{7}\\\cline{1-1}
\end{array}
\le
\begin{array}[c]{*{1}c}\cline{1-1}
\lr{4}\\\cline{1-1}
\lr{5}\\\cline{1-1}
\lr{7}\\\cline{1-1}
\end{array}
\le
\begin{array}[c]{*{1}c}\cline{1-1}
\lr{4}\\\cline{1-1}
\lr{6}\\\cline{1-1}
\lr{7}\\\cline{1-1}
\end{array}
\le
\begin{array}[c]{*{1}c}\cline{1-1}
\lr{5}\\\cline{1-1}
\lr{6}\\\cline{1-1}
\lr{7}\\\cline{1-1}
\end{array}
\]
The matrix \(A'_{\SC}\) is of order \(7 \times 13\) and \(v= (-3,\cdots,-3)\). Hence we get the following matrix \(A_{\SC}=[A'_{\SC}|v]\) of order \(7 \times 14\).
\[
A_{\SC} = 
\left(\begin{array}{rrrrrrrrrrrrrr}
    1 & 1 & 1 & 1 & 1 & 0 & 0 & 0 & 0 & 0 & 0 & 0 & 0 & -3 \\
    1 & 1 & 1 & 1 & 0 & 1 & 1 & 1 & 0 & 0 & 0 & 0 & 0 & -3 \\
    1 & 0 & 0 & 0 & 1 & 1 & 1 & 0 & 1 & 1 & 0 & 0 & 0 & -3 \\
    0 & 1 & 0 & 0 & 0 & 0 & 0 & 1 & 1 & 0 & 1 & 1 & 0 & -3 \\
    0 & 0 & 1 & 0 & 0 & 0 & 0 & 0 & 0 & 1 & 1 & 0 & 1 & -3 \\
    0 & 0 & 0 & 1 & 1 & 1 & 0 & 0 & 0 & 0 & 0 & 1 & 1 & -3 \\
    0 & 0 & 0 & 0 & 0 & 0 & 1 & 1 & 1 & 1 & 1 & 1 & 1 & -3
\end{array}\right)
\]
The monomial indexed by the following tableau is an example of a $T$-invariant polynomial with support in \(\SC\).
\[S = 
    \begin{array}[c]{*{14}c}\cline{1-14}
    \lr{1}&\lr{1}&\lr{1}&\lr{1}&\lr{1}&\lr{1}&\lr{2}&\lr{2}&\lr{2}&\lr{4}&\lr{4}&\lr{4}&\lr{4}&\lr{4}\\\cline{1-14}
    \lr{2}&\lr{2}&\lr{2}&\lr{3}&\lr{3}&\lr{3}&\lr{3}&\lr{3}&\lr{3}&\lr{5}&\lr{5}&\lr{5}&\lr{5}&\lr{5}\\\cline{1-14}
    \lr{4}&\lr{5}&\lr{6}&\lr{6}&\lr{6}&\lr{6}&\lr{6}&\lr{6}&\lr{7}&\lr{7}&\lr{7}&\lr{7}&\lr{7}&\lr{7}\\\cline{1-14}
    \end{array}
    \hspace{4pt}, \hspace{5pt}
    \VEC{v}_{_{S}} = [0, 1, 1, 1, 3, 2, 1, 0, 0, 0, 5, 0, 0, 2].
\]
Observe that \(S\) is degree 14 monomial hence \(\frac{k}{n} = 2 \) which is the last entry in \(v_{_S}\).  


\subsection{Projective normality}
Theorem~\ref{thm:main} and Corollary~\ref{cor:schub} gives us effective ways of computing a generating set of the ring of torus invariants for the Grassmannian and Schubert varieties in the Grassmannian. We were able to run computational experiments in the case of $Gr_{3,7}$.

 \begin{theorem}
 \label{thm:w}
 Let  $w = [3,6,7].$  The line bundle \({\mathcal L}(7\omega_3)\) descends to a line bundle on the GIT quotient \(\LXWmodT{w}{{\cal L}(7\omega_3)}\). The polarised variety \((\LXWmodT{w}{{\cal L}(7\omega_3)},{\cal M})\) is projectively normal.
\end{theorem}
\begin{proof}
The proof is computational. We continue to use the notation from Corollary~\ref{cor:schub}.   Let \({\cal P} \subset I_{3,7}\) be the sub-poset consisting of $v \in I_{r,n}$, $v \leq w$. We verified that there are 252 maximal chains in \({\cal P}\). For each maximal chain $\SC$ we compute $H_{\SC}$, the Hilbert basis of the the pointed cone \(C_{\SC}\). It follows from Corollary~\ref{cor:schub} that the coordinate ring of \(\LXWmodT{\omega}{{\cal L}(7\omega_3)}\) is generated by the union of standard monomials corresponding to elements in $H_{\cal C}$, the union taken over maximal chains  ${\cal C}$ in ${\cal P}$.
    
We observe that the union has 31 standard monomials in degree 7 and 8 standard monomials in degree 14. Using straightening laws, we show that monomials of degree 14 are in algebra generated by monomials in degree 7.  In Appendix~\ref{sec:sch367} we give details of the straightening relations used to straighten the 8 degree 14 monomials.This completes the proof. 
\end{proof}

To compute the Hilbert basis in the example above we used the algorithm given in \cite{hemmecke2002computation}. This algorithm is implemented in package 4ti2 in Sage.

We also compute the Hilbert basis of $T$-invariants in $\oplus_m H^0(\grass{3}{7}, {\cal L}(7m\omega_3))^T$ with support in each maximal chain $\SC$ in $I_{3,7}$.
\eat{
    , using the software package 4ti2 in Sage.  To compute the Hilbert basis we used the algorithm given in \cite{hemmecke2002computation}. Details of that algorithm for the specific problem of computing a generating set of $\oplus_m H^0(\grass{3}{7}, {\cal L}(7m\omega_3))^T$ are described in a companion paper \cite{SS21}.}
Data of that computation is available on github. We have following observations. 
\begin{observation}
\label{obs:gr37}
In $\grass{3}{7}$ the following holds:
\begin{enumerate}
\item There are 462 maximal chains of which 131 have no $T$-invariants.
\item The cardinality of union of standard monomials corresponding to the Hilbert basis of chains $\SC$ in $I_{r,n}$ is 390. There are 225 monomials in degree 7, 157 in degree 14 and 8 in degree 21.
    \eat{degree of a $T$-invariant corresponding to an element in the Hilbert basis of \(\grass{3}{7}\) is \(\eat{3 \times 7=}21\).}
\end{enumerate}
\end{observation}
\eat{
    To get a generating set for the full ring of $T$-invariants  $\oplus_m H^0(\grass{3}{7}, {\cal L}(7m\omega_3))^T$, we take the union of the Hilbert-basis $H_{\SC}$ corresponding to each maximal chain $\SC$ of the Br\"{u}hat poset.}  We have following theorem. 
\begin{theorem}
    The polarised variety \((\Lgmnmodt{3}{7}{\cal L}(7\omega_3),{\cal M})\) is projectively normal, where \(\cal M\) is the descent of \({\mathcal L}(7\omega_3)\) to \(\Lgmnmodt{3}{7}{\cal L}(7\omega_3)\).
\end{theorem}
\begin{proof}
Using straightening laws, we observe that each monomial of degree \(14\) and \(21\) in the generating set of $T$-invariants given by Observation~\ref{obs:gr37} is in the polynomial span of  $T$-invariant monomials  of degree \(7\). So the ring of $T$ invariants is generated in degree $7$. This completes the proof.
\end{proof}

Note that Theorem~\ref{thm:w} follows from the above theorem. This follows because we have a surjective map from $H^0(G/P, {\cal L}(kn\omega_r))$ to 
$H^0(X(\omega), {\cal L}(kn\omega_r))$ for every $k$, which remains surjective after taking $T$ invariants, since $T$ is reductive. Nevertheless, we have given a separate proof since 
we can explicitly carry out the straightening laws in this case and prove $R1$ generation.
 
