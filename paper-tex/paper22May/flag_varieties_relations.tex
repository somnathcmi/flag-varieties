%
\documentclass[11pt]{article}
\usepackage{amsmath,amssymb,amsthm}
\usepackage{graphicx}
\usepackage{datetime}
\usepackage[margin=1in]{geometry}
\usepackage{fancyhdr}
\usepackage{mathtools}
\usepackage{amsmath}
\usepackage{bm}
\usepackage[mathscr]{euscript}
\usepackage{xcolor}
\usepackage{algorithm2e}
\usepackage{enumitem}
\usepackage{stmaryrd}
\renewcommand{\familydefault}{\rmdefault}


%to do: add all possible mathsymbols reference website http://www.peteryu.ca/tutorials/publishing/latex_math_script_styles
\newcommand\MFb{\mathfrak{b}}
\newcommand\MFt{\mathfrak{t}}
\newcommand\MFw{\mathfrak{w}}
\newcommand\IR{\mathbb{R}}
\newcommand\IN{\mathbb{N}}
\newcommand\IZ{\mathbb{Z}}
\newcommand\IC{\mathbb{C}}
\newcommand\IQ{\mathbb{Q}}
\newcommand\IF{\mathbb{F}}
\newcommand\SF{\mathscr{F}}
\newcommand\SC{\mathscr{C}}
\newcommand\SH{\mathscr{H}}
\newcommand\SB{\mathscr{B}}
\newcommand\SP{\mathscr{P}}
\newcommand\POWSET{\SP}%power set symbol
\newcommand\BF{\textbf}
\newcommand\IT{\textit}
\newcommand\UL{\underline}
\newcommand\DUL{\underline\underline}
\newcommand\OL[1]{$\overline{\mbox{#1}}$}
\newcommand\ITBF[1]{\textbf{\textit{#1}}}
\newcommand\ITUL[1]{\underline{\textit{#1}}}
\newcommand\BFUL[1]{\textbf{\underline{#1}}}
\newcommand\ITBFUL[1]{\textbf{\textit{\underline{#1}}}}
\newcommand\VEC{\bm}
\newcommand\GIVEN{\mid}%for conditional probabilities
\newcommand\UNI{u}%symbol for uniform random smapeling
\newcommand\DEF{:=}

\newcommand\bigo[1]{\mathcal{O}(#1)}

\newcommand\comment[1]{\footnote{\color{red}#1}}
\newcommand\npara{
\setlength{\parindent}{0ex}
\setlength{\parskip}{0.5em}
}
\newcommand\thmpara{
\setlength{\parindent}{0ex}
\setlength{\parskip}{0em}
}

\newcommand\BI[5]{
\bibitem[#1]{#1}
    #2.
    \newblock {\em #3}.
    \newblock #4.
    \newblock {\em #5}.
}

\pagestyle{fancyplain}
\chead{}
\lhead{}
\rhead{}
\rfoot{}

\newtheorem{theorem}{Theorem}
\newtheorem{lemma}[theorem]{Lemma}
\newtheorem{corollary}[theorem]{Corollary}
\newtheorem{definition}[theorem]{Definition}
\newtheorem{claim}[theorem]{Claim}
\newtheorem{fact}[theorem]{Fact}
\newtheorem{remark}[theorem]{Remark}
\newtheorem{problem}[theorem]{Problem}

\allowdisplaybreaks


%\begin{document}

\section{Flag varieties whose $T$ quotients are Grassmannians}
\subsection{Shape \((r,1^s)\)}
\eat{
We have following lemma
\begin{lemma}
    Let \(S\in STab(\lambda,n,\SC)\) of degree \(d\) 
    where \(\SC\) is maximal chain in Bruhat poset \((I(\lambda,n), \le)\).
    For any column \(\UL{p}\) of length \(r\) in \(S\), 
    there is factor of \(S\) of degree \(1\) which contain \(\UL{p}\).
\end{lemma}
\begin{proof}
    Shape and weight of \(S\) are \((r^d,1^{sd})\) and \(d\VEC{1}\) respectively.
    Let \(S = X_1X_2\) such that shape of \(X_1\) is \((r^d)\) and shape of \(X_2\) is \((1^{sd})\).
    Let \(\UL{p}\) be a column in \(X_1\) and \(i \in [n]\) such that 
    \(i\) do not appear in \(\UL{p}\). We have that weight of \(i\) in \(X_1\) is at most \(d-1\)
    (There are exactly \(d\) columns in \(X_1\) and entries in colums are distinct). 
    This implies weight of \(i\) in \(X_2\) is at least \(1\) 
    (weight of \(i\) in \(S=X_1X_2\) is \(d\)).

    Factor of \(S\) of degree \(1\) is constructed by taking any column \(p\) in \(X_1\) 
    and all \(i'\)s not appearing in \(\UL{p}\) from \(X_2\).
\end{proof}
\begin{corollary}
    Flag variety \(\SF\SL_{\lambda}\) is projectively normal.\qed
\end{corollary}
In this subsection we will assume that \(\lambda = (r,1^s)\) and \(r+s=n\). 
We define linear map \(\phi:\IC[\SF\SL_{\lambda}]^T \to \IC[\grass{n-1}{r-1}]\) as follows: \(\phi_d \DEF \phi|_{\IC[\SF\SL_{\lambda}]_d}\) is
\begin{align*}
    S=
    \begin{array}[c]{*{6}c}\cline{1-6}
        \lr{S_{11}}&\lr{\cdots}&\lr{S_{1d}}&\lr{S_{1(d+1)}}&\lr{\cdots}&\lr{S_{1(d+sd)}}\\\cline{1-6}
        \lr{S_{21}}&\lr{\cdots}&\lr{S_{2d}}\\\cline{1-3}
        \lr{\vdots}&\lr{\vdots}&\lr{\vdots}\\\cline{1-3}
        \lr{S_{r1}}&\lr{\cdots}&\lr{S_{rd}}\\\cline{1-3}
    \end{array}
    &\mapsto S'=
    \begin{array}[c]{*{3}c}\cline{1-3}
        \lr{S_{21}}&\lr{\cdots}&\lr{S_{2d}}\\\cline{1-3}
        \lr{\vdots}&\lr{\vdots}&\lr{\vdots}\\\cline{1-3}
        \lr{S_{r1}}&\lr{\cdots}&\lr{S_{rd}}\\\cline{1-3}
    \end{array}
\end{align*}
Tableau \(S'\) is obtained by removing first row in tableau \(S\). Weight of \(1\) in tableau \(S\) is \(d\) hence weight of \(1\) in \(S'\) is \(0\). We get \(\phi_d\) is isomorphism of vector spaces. 
}
Let $G=SL(n,{\mathbb C})$. We look at flag varieties of the form $G/Q$, where $Q$ is the intersection of two maximal parabolic subgroups, as in Section~\ref{sec:fv_normality}. 
We show that for certain $Q$ and a chosen polarization that the $T$-quotients are Grassmannians.
 
Let $r$ be a positive integer such that $1 \leq r \leq n-1$ and set $s = n-r$.  Let  $Q = P^{\alpha_r} \cap P^{\alpha_1}$. Let $\lambda = \omega_r + s\omega_1$.
Let \[I'_{r,n} = \{{\UL i} \in I_{r,n} : i_1 = 1\}.\]
\[I'_{r-1,n-1} = \{(i_2,\cdots,i_r):2 \le i_2 < \cdots < i_r \le n\}.\]
The indexing set for  Pl\"{u}cker coordinates on \(\grass{r-1}{n-1}\) is $[1 \leq i_1 < i_2 < \cdots < i_{r-1} \leq n-1]$, and this set can be seen to be in bijection with the set $I'_{r-1,n-1}$ by the map sending 
$[i_1,i_2,\ldots,i_{r-1}]$ to $[i_1 + 1, i_2 + 1,\ldots, i_{r-1}+1]$.  Let \(\phi'\) be natural bijection between \(I'_{r,n}\) and \(I'_{r-1,n-1}\), sending $[1, i_2,\ldots,i_n]$ to $[i_2,\ldots,i_n]$.
Composing this bijection with the inverse of the earlier bijection, we have the bijection
$$\phi'[1, i_2,i_3,\ldots, i_r] \mapsto [i_2-1,i_3-1,\ldots,i_r - 1].$$

Note that \(\oplus H^0(\gmodq,{\cal L}(\lambda)^{\otimes k})\) is generated by standard monomials of shape \((r^k,1^{sk})^t\) for \(k\) positive integer. Here $(,)^t$ denotes taking the transpose shape.  In the rest of this section we will identify standard monomials with the semistandard tableaux associated to them.

Recall from Proposition~\ref{prop:tinv} that a standard monomial is $T$ invariant only when its weight is uniform. We have following observation.

\begin{observation}
    \(p_{_X} = p_{{\UL i}_1}p_{{\UL i}_2}\cdots p_{{\UL i}_k} p_{i'_1}p_{i'_2}\cdots p_{i'_{sk}} \in H^0(\gmodq,{\cal L}(\lambda)^{\otimes k})^T\) be a standard monomial, where \({\UL i}_1,\cdots,{\UL i}_k \in I_{r,n}\) and \(i'_1,\cdots, i'_{sk} \in I_{1,n} = [n]\).
\begin{enumerate}
    \item \(wt(p_{_X}) = (k,\cdots,k)\).
    \item \(X_{1,1} = X_{1,2} = \cdots = X_{1,k} = 1\), i.e. first entry in columns of length \(r\) is 1.
\end{enumerate}
\end{observation}
\begin{proof}
    Since the shape of \(X\) is \((r^k,1^{ks})^t\), the number of boxes in \(X\) is \(kr+ks=kn\). Since \(p_{_X}\) is \(T\)-invariant each integer in \([n]\) appears an equal number of times and \(wt(X) = (k,\cdots,k)\). Since \(p_{_X}\) is a standard monomial the first \(k\) entries of first row in $X$ must be \(1\).
\end{proof}
With the help of above observation, the bijection from \(\phi':I'_{r,n}\to I_{r-1,n-1}\) can be extended to a ${\mathbb C}$-linear map \(\phi:H^0(\gmodq,{\cal L}(\lambda)^{\otimes k})^T \to H^0(G'/P',{\cal L}(\omega_{r-1})^{\otimes k})\), where \(G'/P'\) is Grassmannian \(\grass{r-1}{n-1}\), as follows:
    \[\phi(p_{{\UL i}_1}p_{{\UL i}_2}\cdots p_{{\UL i}_k} p_{i'_1}p_{i'_2}\cdots p_{i'_{sk}})
    = p_{\phi'({\UL i}_1)}p_{\phi'({\UL i}_2)}\cdots p_{\phi'({\UL i}_k)}\] 
Noe that the map \(\phi\) is well defined. We show that \(\phi\) is ring map.
\begin{lemma}
    Let \(\phi\) be map as above and \(X,Y\) be tableaux such that
    \(p_{_X} \in H^0(\gmodq,{\cal L}(\lambda)^{\otimes k})^T\) 
    and \(p_{_Y} \in H^0(\gmodq,{\cal L}(\lambda)^{\otimes k'})^T\). 
    We have
    \begin{enumerate}
        \item \(\phi\) is vector space isomorphism.
        \item \(\phi(p_{_X}p_{_Y}) = \phi(p_{_X})\phi(p_{_Y})\).
    \end{enumerate}
    Hence \(\phi\) is a \(\IC\)-algebra isomorphism.
\end{lemma}
\begin{proof}
    Observe that the tableau corresponding to the standard monomial \(\phi(p_{_X})\) is obtained by removing the first row of the tableau \(X\) and decreasing every entry of the resulting rectangular tableau by 1. We claim this is a bijection. For every standard monomial \(p_{_Z}\) in \(H^0(G'/P', {\cal L}(\omega_{r-1})^{\otimes k})\), we claim that there is unique semistandard monomial \(p_{_X}\) in \(H^0(G/Q,{\cal L}(\lambda)^{\otimes k})^T\) such that \(\phi(p_{_X})=p_{_Z}\). Start with the tableau $Z$ corresponding to  \(p_{_Z}\). Increase each entry of $Z$ by 1, to get a semistandard rectangular tableau $Z'$ with entries in $[2,\ldots,n]$. Let the weight of $Z'$ be $wt(Z')$.  We construct the tableau \(X\) as follows: construct the semistandard tableau \(X_{1,*}\) of weight \(v = (k,k,...,k) - wt(Z')\), having exactly one row. Note that this has at least $k$ columns and the entries in the first $k$ columns are all 1's. Tableau \(X\) is constructed by appending the tableau \(Z'\) below \(X_{1,*}\). Clearly $X$ is a semistandard tableau. It follows from the construction that \(p_{_X}\) is \(T\)-invariant standard monomial. Clearly, \(\phi(p_{_X}) = p_{_Z}\).

2. Let $X_1, X_2$ be the semistandard tableau corresponding to $T$-invariant monomials $p_{X_1}$, $p_{X_2}$.  If the product $p_{X_1}p_{X_2}$ is standard, the columns of
$X_1, X_2$ can be rearranged so that the tableau $X$ corresponding to the product monomial is semistandard. Note that $X$ has $2k$ columns of
length $r$ and $2sk$ columns of length 1. The image of $p_X$ is the monomial corresponding to the tableau (of size $r-1 \times 2k$ )obtained from $X$ by taking the subtableau of $X$ of size $r \times 2k$, removing the first row and then decreasing every entry of the resulting tableau. But this is precisely the tableau corresponding to the product of
the monomials $\phi(p_{X_1})$, $\phi(p_{X_2})$.

If $p_{X_1}p_{X_2}$ is non standard, we can express it as the sum of products of 
standard monomials. Note that the straightening relations needed to straighten the product only involves columns of length $r$ in $X_1$, $X_2$. Now, the first rows of both $X_1$ and $X_2$ have 1's. So this row remains untouched during straightening. Recall $\phi$ is a bijection. If a straightening relation is applied on the left hand side to straighten two columns, we can apply the same straightening relation to the images of these columns under $\phi$ on the right hand side.  This completes the proof.
\end{proof}
We have,
\begin{theorem}
Let $(r,n)=1$, and let $s=n-r$. Set $Q = P^{\alpha_r} \cap P^{\alpha_1}$ and let $\lambda = \omega_r + s\omega_1$. Consider the embedding of $G/Q$ given by the line bundle ${\mathcal L}(\lambda)$. Let $X = \nflvmodt {\cal L}(\lambda)$ be the GIT quotient. The line bundle ${\mathcal L}(\lambda)$ descends to an ample line bundle ${\cal M}$ on $X$, and the polarized variety $(X,{\cal M})$ is isomorphic to $Gr_{r-1, n-1}$.

\end{theorem}

%\subsection{Shape \((3,2)\)}

%\subsection{Shape \((2,2,1)\)}

%\end{document}
