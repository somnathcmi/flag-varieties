\section{Generators of \(T\) quotient of partial Flag varieties.}
\label{sec:fv_hb}

In this section we will present the calculations leading to Theorem~\ref{thm:flag} for \(\lambda=a\omega_r+b\omega_s\) such that \(ar+bs = 5\). We have following possible values of \(\lambda\): 1. \(\omega_3+\omega_2\), 2. \(\omega_4+\omega_1\), 3. \(\omega_3+2\omega_1\), 4. \(2\omega_2+\omega_1\) and 5. \(\omega_2+3\omega_1\) We have dealt the first case in Section ~\ref{sec:fv_normality} where we showed calculations in one chain in poset \(I_{\lambda,5}\) for \(\lambda = \omega_3+\omega_2\).  The only details missing in that proof were about the polynomials $f_1, f_2$. We describe those polynomials first in Section~\ref{sec:fv_hb40}. The other cases are actually easier.
\subsection{$\lambda=\omega_3+\omega_2$}
\label{sec:fv_hb40}
The polynomial \(f_1\) is obtained by straightening the monomial \(x_1x_2\)
\[
x_1x_2 = 
\begin{array}[c]{*{8}c}\cline{1-8}
\lr{1}&\lr{1}&\lr{1}&\lr{1}&\lr{2}&\lr{2}&\lr{3}&\lr{4}\\\cline{1-8}
\lr{2}&\lr{2}&\lr{3}&\lr{4}&\lr{3}&\lr{4}&\lr{5}&\lr{5}\\\cline{1-8}
\lr{3}&\lr{4}&\lr{5}&\lr{5}\\\cline{1-4}
\end{array}
\]
Observe that the monomial is nonstandard and non-comparable pair is \([1,4,5],[2,3]\). We will use following Pl\'{u}cker relation.
\[
\begin{array}[c]{*{2}c}\cline{1-2}
\lr{1}&\lr{2}\\\cline{1-2}
\lr{4}&\lr{3}\\\cline{1-2}
\lr{5}\\\cline{1-1}
\end{array}
-\begin{array}[c]{*{2}c}\cline{1-2}
\lr{1}&\lr{2}\\\cline{1-2}
\lr{3}&\lr{4}\\\cline{1-2}
\lr{5}\\\cline{1-1}
\end{array}
+\begin{array}[c]{*{2}c}\cline{1-2}
\lr{1}&\lr{3}\\\cline{1-2}
\lr{2}&\lr{4}\\\cline{1-2}
\lr{5}\\\cline{1-1}
\end{array}
+\begin{array}[c]{*{2}c}\cline{1-2}
\lr{1}&\lr{2}\\\cline{1-2}
\lr{3}&\lr{5}\\\cline{1-2}
\lr{4}\\\cline{1-1}
\end{array}
-\begin{array}[c]{*{2}c}\cline{1-2}
\lr{1}&\lr{3}\\\cline{1-2}
\lr{2}&\lr{5}\\\cline{1-2}
\lr{4}\\\cline{1-1}
\end{array}
+\begin{array}[c]{*{2}c}\cline{1-2}
\lr{1}&\lr{4}\\\cline{1-2}
\lr{2}&\lr{5}\\\cline{1-2}
\lr{3}\\\cline{1-1}
\end{array}
=0
\]
We get
\begin{align*}
x_1x_2 
& = 
\begin{array}[c]{*{8}c}\cline{1-8}
\lr{1}&\lr{1}&\lr{1}&\lr{1}&\lr{2}&\lr{2}&\lr{3}&\lr{4}\\\cline{1-8}
\lr{2}&\lr{2}&\lr{3}&\lr{4}&\lr{3}&\lr{4}&\lr{5}&\lr{5}\\\cline{1-8}
\lr{3}&\lr{4}&\lr{5}&\lr{5}\\\cline{1-4}
\end{array}\\
&=\begin{array}[c]{*{8}c}\cline{1-8}
\lr{1}&\lr{1}&\lr{1}&\lr{1}&\lr{2}&\lr{2}&\lr{3}&\lr{4}\\\cline{1-8}
\lr{2}&\lr{2}&\lr{3}&\lr{3}&\lr{4}&\lr{4}&\lr{5}&\lr{5}\\\cline{1-8}
\lr{3}&\lr{4}&\lr{5}&\lr{5}\\\cline{1-4}
\end{array}
-\begin{array}[c]{*{8}c}\cline{1-8}
\lr{1}&\lr{1}&\lr{1}&\lr{1}&\lr{2}&\lr{3}&\lr{3}&\lr{4}\\\cline{1-8}
\lr{2}&\lr{2}&\lr{2}&\lr{3}&\lr{4}&\lr{4}&\lr{5}&\lr{5}\\\cline{1-8}
\lr{3}&\lr{4}&\lr{5}&\lr{5}\\\cline{1-4}
\end{array}
-\begin{array}[c]{*{8}c}\cline{1-8}
\lr{1}&\lr{1}&\lr{1}&\lr{1}&\lr{2}&\lr{2}&\lr{3}&\lr{4}\\\cline{1-8}
\lr{2}&\lr{2}&\lr{3}&\lr{3}&\lr{4}&\lr{5}&\lr{5}&\lr{5}\\\cline{1-8}
\lr{3}&\lr{4}&\lr{4}&\lr{5}\\\cline{1-4}
\end{array}\\
&\hspace{200pt}
+\begin{array}[c]{*{8}c}\cline{1-8}
\lr{1}&\lr{1}&\lr{1}&\lr{1}&\lr{2}&\lr{3}&\lr{3}&\lr{4}\\\cline{1-8}
\lr{2}&\lr{2}&\lr{2}&\lr{3}&\lr{4}&\lr{5}&\lr{5}&\lr{5}\\\cline{1-8}
\lr{3}&\lr{4}&\lr{4}&\lr{5}\\\cline{1-4}
\end{array}
-\begin{array}[c]{*{8}c}\cline{1-8}
\lr{1}&\lr{1}&\lr{1}&\lr{1}&\lr{2}&\lr{3}&\lr{4}&\lr{4}\\\cline{1-8}
\lr{2}&\lr{2}&\lr{2}&\lr{3}&\lr{4}&\lr{5}&\lr{5}&\lr{5}\\\cline{1-8}
\lr{3}&\lr{3}&\lr{4}&\lr{5}\\\cline{1-4}
\end{array}\\
& \\
&=
t_1t_0t_3^2 - t_1t_0t_2t_3 - t_1t_0t_4t_3 + t_1t_0^2t_3 - t_1^2t_0t_3.
\end{align*}
Hence we define,
\begin{align*}
f_1(t_0,t_1,t_2,t_3,t_4) &\DEF x_1x_2\\
& = t_1t_0t_3^2 - t_1t_0t_2t_3 - t_1t_0t_4t_3 + t_1t_0^2t_3 - t_1^2t_0t_3.
\end{align*}

\subsubsection{Polynomial \(f_2\)}
To get \(f_2\) we will straigntain following monomial
\[
t_2t_4 = 
\begin{array}[c]{*{4}c}\cline{1-4}
\lr{1}&\lr{1}&\lr{2}&\lr{3}\\\cline{1-4}
\lr{2}&\lr{3}&\lr{5}&\lr{4}\\\cline{1-4}
\lr{5}&\lr{4}\\\cline{1-2}
\end{array}
\]
We will use following Pl\'{u}cker relations
\begin{align*}
\begin{array}[c]{*{2}c}\cline{1-2}
\lr{1}&\lr{1}\\\cline{1-2}
\lr{2}&\lr{3}\\\cline{1-2}
\lr{5}&\lr{4}\\\cline{1-2}
\end{array}
+\begin{array}[c]{*{2}c}\cline{1-2}
\lr{1}&\lr{1}\\\cline{1-2}
\lr{2}&\lr{4}\\\cline{1-2}
\lr{3}&\lr{5}\\\cline{1-2}
\end{array}
-\begin{array}[c]{*{2}c}\cline{1-2}
\lr{1}&\lr{1}\\\cline{1-2}
\lr{2}&\lr{3}\\\cline{1-2}
\lr{4}&\lr{5}\\\cline{1-2}
\end{array}
&=0 &&\cdots (\text{R1})\\
\begin{array}[c]{*{2}c}\cline{1-2}
\lr{2}&\lr{3}\\\cline{1-2}
\lr{5}&\lr{4}\\\cline{1-2}
\end{array}
-\begin{array}[c]{*{2}c}\cline{1-2}
\lr{2}&\lr{3}\\\cline{1-2}
\lr{4}&\lr{5}\\\cline{1-2}
\end{array}
+\begin{array}[c]{*{2}c}\cline{1-2}
\lr{2}&\lr{4}\\\cline{1-2}
\lr{3}&\lr{5}\\\cline{1-2}
\end{array}
&=0 &&\cdots (\text{R2})\\
\begin{array}[c]{*{2}c}\cline{1-2}
\lr{1}&\lr{2}\\\cline{1-2}
\lr{4}&\lr{3}\\\cline{1-2}
\lr{5}\\\cline{1-1}
\end{array}
-\begin{array}[c]{*{2}c}\cline{1-2}
\lr{1}&\lr{2}\\\cline{1-2}
\lr{3}&\lr{4}\\\cline{1-2}
\lr{5}\\\cline{1-1}
\end{array}
+\begin{array}[c]{*{2}c}\cline{1-2}
\lr{1}&\lr{3}\\\cline{1-2}
\lr{2}&\lr{4}\\\cline{1-2}
\lr{5}\\\cline{1-1}
\end{array}
+\begin{array}[c]{*{2}c}\cline{1-2}
\lr{1}&\lr{2}\\\cline{1-2}
\lr{3}&\lr{5}\\\cline{1-2}
\lr{4}\\\cline{1-1}
\end{array}
-\begin{array}[c]{*{2}c}\cline{1-2}
\lr{1}&\lr{3}\\\cline{1-2}
\lr{2}&\lr{5}\\\cline{1-2}
\lr{4}\\\cline{1-1}
\end{array}
+\begin{array}[c]{*{2}c}\cline{1-2}
\lr{1}&\lr{4}\\\cline{1-2}
\lr{2}&\lr{5}\\\cline{1-2}
\lr{3}\\\cline{1-1}
\end{array}
&=0 &&\cdots (\text{R3})
\end{align*}
We get
\begin{align*}
t_2t_4 &= 
\begin{array}[c]{*{4}c}\cline{1-4}
\lr{1}&\lr{1}&\lr{2}&\lr{3}\\\cline{1-4}
\lr{2}&\lr{3}&\lr{5}&\lr{4}\\\cline{1-4}
\lr{5}&\lr{4}\\\cline{1-2}
\end{array}\\
&=
-\begin{array}[c]{*{4}c}\cline{1-4}
\lr{1}&\lr{1}&\lr{2}&\lr{3}\\\cline{1-4}
\lr{2}&\lr{4}&\lr{5}&\lr{4}\\\cline{1-4}
\lr{3}&\lr{5}\\\cline{1-2}
\end{array}
+\begin{array}[c]{*{4}c}\cline{1-4}
\lr{1}&\lr{1}&\lr{2}&\lr{3}\\\cline{1-4}
\lr{2}&\lr{3}&\lr{5}&\lr{4}\\\cline{1-4}
\lr{4}&\lr{5}\\\cline{1-2}
\end{array}
&&\cdots (\text{After applying R1})
\\
&=
\begin{array}[c]{*{4}c}\cline{1-4}
\lr{1}&\lr{1}&\lr{2}&\lr{3}\\\cline{1-4}
\lr{2}&\lr{3}&\lr{5}&\lr{4}\\\cline{1-4}
\lr{4}&\lr{5}\\\cline{1-2}
\end{array}
-\begin{array}[c]{*{4}c}\cline{1-4}
\lr{1}&\lr{1}&\lr{2}&\lr{3}\\\cline{1-4}
\lr{2}&\lr{4}&\lr{4}&\lr{5}\\\cline{1-4}
\lr{3}&\lr{5}\\\cline{1-2}
\end{array}
+\begin{array}[c]{*{4}c}\cline{1-4}
\lr{1}&\lr{1}&\lr{2}&\lr{4}\\\cline{1-4}
\lr{2}&\lr{4}&\lr{3}&\lr{5}\\\cline{1-4}
\lr{3}&\lr{5}\\\cline{1-2}
\end{array}
&&\cdots (\text{Term 1, Apply R2})
\\
&=
-\begin{array}[c]{*{4}c}\cline{1-4}
\lr{1}&\lr{1}&\lr{2}&\lr{3}\\\cline{1-4}
\lr{2}&\lr{4}&\lr{4}&\lr{5}\\\cline{1-4}
\lr{3}&\lr{5}\\\cline{1-2}
\end{array}
+\begin{array}[c]{*{4}c}\cline{1-4}
\lr{1}&\lr{1}&\lr{2}&\lr{4}\\\cline{1-4}
\lr{2}&\lr{4}&\lr{3}&\lr{5}\\\cline{1-4}
\lr{3}&\lr{5}\\\cline{1-2}
\end{array}
+\begin{array}[c]{*{4}c}\cline{1-4}
\lr{1}&\lr{1}&\lr{2}&\lr{3}\\\cline{1-4}
\lr{2}&\lr{3}&\lr{4}&\lr{5}\\\cline{1-4}
\lr{4}&\lr{5}\\\cline{1-2}
\end{array}
-\begin{array}[c]{*{4}c}\cline{1-4}
\lr{1}&\lr{1}&\lr{2}&\lr{4}\\\cline{1-4}
\lr{2}&\lr{3}&\lr{3}&\lr{5}\\\cline{1-4}
\lr{4}&\lr{5}\\\cline{1-2}
\end{array}
&&\cdots (\text{Term 1, Apply R2})
\\
&=
-\begin{array}[c]{*{4}c}\cline{1-4}
\lr{1}&\lr{1}&\lr{2}&\lr{3}\\\cline{1-4}
\lr{2}&\lr{4}&\lr{4}&\lr{5}\\\cline{1-4}
\lr{3}&\lr{5}\\\cline{1-2}
\end{array}
+\begin{array}[c]{*{4}c}\cline{1-4}
\lr{1}&\lr{1}&\lr{2}&\lr{3}\\\cline{1-4}
\lr{2}&\lr{3}&\lr{4}&\lr{5}\\\cline{1-4}
\lr{4}&\lr{5}\\\cline{1-2}
\end{array}
-\begin{array}[c]{*{4}c}\cline{1-4}
\lr{1}&\lr{1}&\lr{2}&\lr{4}\\\cline{1-4}
\lr{2}&\lr{3}&\lr{3}&\lr{5}\\\cline{1-4}
\lr{4}&\lr{5}\\\cline{1-2}
\end{array}
+\begin{array}[c]{*{4}c}\cline{1-4}
\lr{1}&\lr{1}&\lr{2}&\lr{4}\\\cline{1-4}
\lr{2}&\lr{3}&\lr{4}&\lr{5}\\\cline{1-4}
\lr{3}&\lr{5}\\\cline{1-2}
\end{array}
\\
&\hspace{70pt}
-\begin{array}[c]{*{4}c}\cline{1-4}
\lr{1}&\lr{1}&\lr{3}&\lr{4}\\\cline{1-4}
\lr{2}&\lr{2}&\lr{4}&\lr{5}\\\cline{1-4}
\lr{3}&\lr{5}\\\cline{1-2}
\end{array}
-\begin{array}[c]{*{4}c}\cline{1-4}
\lr{1}&\lr{1}&\lr{2}&\lr{4}\\\cline{1-4}
\lr{2}&\lr{3}&\lr{5}&\lr{5}\\\cline{1-4}
\lr{3}&\lr{4}\\\cline{1-2}
\end{array}
+\begin{array}[c]{*{4}c}\cline{1-4}
\lr{1}&\lr{1}&\lr{3}&\lr{4}\\\cline{1-4}
\lr{2}&\lr{2}&\lr{5}&\lr{5}\\\cline{1-4}
\lr{3}&\lr{4}\\\cline{1-2}
\end{array}
-\begin{array}[c]{*{4}c}\cline{1-4}
\lr{1}&\lr{1}&\lr{4}&\lr{4}\\\cline{1-4}
\lr{2}&\lr{2}&\lr{5}&\lr{5}\\\cline{1-4}
\lr{3}&\lr{3}\\\cline{1-2}
\end{array}
\\
& &&\cdots (\text{Term 2, Apply R3})\\
&=
-x_2 + t_0t_3 - x_1 + t_1t_3 - t_1t_2 - t_1t_4 + t_1t_0 - t_1^2.
\end{align*}

Define \(f_2\) as follows:
\begin{align*}
    f_2(t_0,t_1,t_2,t_3,t_4) &\DEF x_1 + x_2\\ 
    & = - t_2t_4 + t_0t_3 + t_1t_3 - t_1t_2 - t_1t_4 + t_1t_0 - t_1^2.
\end{align*}


\subsection{\(\lambda=\omega_4+\omega_1\)}
\label{sec:fv_hb41}
The tableaux corresponding to the generators of the ring of $T$-invariants obtained from the union of Hilbert basis $H_{\SC}$ are,
\[
 \begin{array}[c]{*{2}c}\cline{1-2}
 \lr{1}&\lr{5}\\\cline{1-2}
 \lr{2}\\\cline{1-1}
 \lr{3}\\\cline{1-1}
 \lr{4}\\\cline{1-1}
 \end{array},\hspace{20pt}
 \begin{array}[c]{*{2}c}\cline{1-2}
 \lr{1}&\lr{4}\\\cline{1-2}
 \lr{2}\\\cline{1-1}
 \lr{3}\\\cline{1-1}
 \lr{5}\\\cline{1-1}
 \end{array},\hspace{20pt}
 \begin{array}[c]{*{2}c}\cline{1-2}
 \lr{1}&\lr{3}\\\cline{1-2}
 \lr{2}\\\cline{1-1}
 \lr{4}\\\cline{1-1}
 \lr{5}\\\cline{1-1}
 \end{array},\hspace{20pt}
 \begin{array}[c]{*{2}c}\cline{1-2}
 \lr{1}&\lr{2}\\\cline{1-2}
 \lr{3}\\\cline{1-1}
 \lr{4}\\\cline{1-1}
 \lr{5}\\\cline{1-1}
 \end{array}
 \]

Hence ring of invariants is generated by first graded component.

\subsection{\(\lambda=\omega_3+2\omega_1\)}
\label{sec:fv_hb311}
The tableaux corresponding to the generators of the ring of $T$-invariants obtained from the union of Hilbert basis $H_{\SC}$ are
\[
 \begin{array}[c]{*{3}c}\cline{1-3}
 \lr{1}&\lr{4}&\lr{5}\\\cline{1-3}
 \lr{2}\\\cline{1-1}
 \lr{3}\\\cline{1-1}
 \end{array},\hspace{20pt}
 \begin{array}[c]{*{3}c}\cline{1-3}
 \lr{1}&\lr{3}&\lr{5}\\\cline{1-3}
 \lr{2}\\\cline{1-1}
 \lr{4}\\\cline{1-1}
 \end{array},\hspace{20pt}
 \begin{array}[c]{*{3}c}\cline{1-3}
 \lr{1}&\lr{3}&\lr{4}\\\cline{1-3}
 \lr{2}\\\cline{1-1}
 \lr{5}\\\cline{1-1}
 \end{array},\hspace{20pt}
 \begin{array}[c]{*{3}c}\cline{1-3}
 \lr{1}&\lr{2}&\lr{5}\\\cline{1-3}
 \lr{3}\\\cline{1-1}
 \lr{4}\\\cline{1-1}
 \end{array},\hspace{20pt}
 \begin{array}[c]{*{3}c}\cline{1-3}
 \lr{1}&\lr{2}&\lr{4}\\\cline{1-3}
 \lr{3}\\\cline{1-1}
 \lr{5}\\\cline{1-1}
 \end{array},\hspace{20pt}
 \begin{array}[c]{*{3}c}\cline{1-3}
 \lr{1}&\lr{2}&\lr{3}\\\cline{1-3}
 \lr{4}\\\cline{1-1}
 \lr{5}\\\cline{1-1}
 \end{array}
 \]


The ring of invariants is generated by the first graded component.

\subsection{\(\lambda = 2\omega_2+\omega_1\)}
\label{sec:fv_hb221}
In this case, the tableaux corresponding to the generators of the ring of $T$-invariants obtained from the union of Hilbert basis $H_{\SC}$ are
\[
t_{0}=\begin{array}[c]{*{3}c}\cline{1-3}
\lr{1}&\lr{2}&\lr{3}\\\cline{1-3}
\lr{4}&\lr{5}\\\cline{1-2}
\end{array}
\hspace{20px}
t_{1}=\begin{array}[c]{*{3}c}\cline{1-3}
\lr{1}&\lr{2}&\lr{4}\\\cline{1-3}
\lr{3}&\lr{5}\\\cline{1-2}
\end{array}
\hspace{20px}
t_{2}=\begin{array}[c]{*{3}c}\cline{1-3}
\lr{1}&\lr{3}&\lr{4}\\\cline{1-3}
\lr{2}&\lr{5}\\\cline{1-2}
\end{array}
\hspace{20px}
t_{3}=\begin{array}[c]{*{3}c}\cline{1-3}
\lr{1}&\lr{3}&\lr{5}\\\cline{1-3}
\lr{2}&\lr{4}\\\cline{1-2}
\end{array}
\hspace{20px}
t_{4}=\begin{array}[c]{*{3}c}\cline{1-3}
\lr{1}&\lr{2}&\lr{5}\\\cline{1-3}
\lr{3}&\lr{4}\\\cline{1-2}
\end{array}
\]
\[
\hspace{20px}
x_{1}=\begin{array}[c]{*{6}c}\cline{1-6}
\lr{1}&\lr{1}&\lr{2}&\lr{3}&\lr{3}&\lr{5}\\\cline{1-6}
\lr{2}&\lr{4}&\lr{4}&\lr{5}\\\cline{1-4}
\end{array}
\hspace{20px}
x_{2}=\begin{array}[c]{*{6}c}\cline{1-6}
\lr{1}&\lr{1}&\lr{2}&\lr{4}&\lr{4}&\lr{5}\\\cline{1-6}
\lr{2}&\lr{3}&\lr{3}&\lr{5}\\\cline{1-4}
\end{array}
\hspace{20px}
\]

We have the following lemma.
\begin{lemma}
    \(x_1,x_2 \in \IC[t_0,\cdots,t_4]\) using straightening laws. Hence the Krull dimension of \(\IC[t_0,\cdots,t_4,x_1,x_2]\) is 5
\end{lemma}
\begin{proof}
    We have the following relations:
    \begin{align*}
t_0t_3 &= 
\begin{array}[c]{*{6}c}\cline{1-6}
\lr{1}&\lr{1}&\lr{2}&\lr{3}&\lr{3}&\lr{5}\\\cline{1-6}
\lr{2}&\lr{4}&\lr{5}&\lr{4}\\\cline{1-4}
\end{array}\\
&=
\begin{array}[c]{*{6}c}\cline{1-6}
\lr{1}&\lr{1}&\lr{2}&\lr{3}&\lr{3}&\lr{5}\\\cline{1-6}
\lr{2}&\lr{4}&\lr{4}&\lr{5}\\\cline{1-4}
\end{array}
-\begin{array}[c]{*{6}c}\cline{1-6}
\lr{1}&\lr{1}&\lr{2}&\lr{3}&\lr{4}&\lr{5}\\\cline{1-6}
\lr{2}&\lr{3}&\lr{4}&\lr{5}\\\cline{1-4}
\end{array}
+\begin{array}[c]{*{6}c}\cline{1-6}
\lr{1}&\lr{1}&\lr{2}&\lr{3}&\lr{5}&\lr{5}\\\cline{1-6}
\lr{2}&\lr{3}&\lr{4}&\lr{4}\\\cline{1-4}
\end{array}
+\begin{array}[c]{*{6}c}\cline{1-6}
\lr{1}&\lr{1}&\lr{3}&\lr{3}&\lr{4}&\lr{5}\\\cline{1-6}
\lr{2}&\lr{2}&\lr{4}&\lr{5}\\\cline{1-4}
\end{array}
-\begin{array}[c]{*{6}c}\cline{1-6}
\lr{1}&\lr{1}&\lr{3}&\lr{3}&\lr{5}&\lr{5}\\\cline{1-6}
\lr{2}&\lr{2}&\lr{4}&\lr{4}\\\cline{1-4}
\end{array}\\
&= x_1 - t_2t_4 + t_3t_4 + t_2t_3 - t_3^2\\
t_1t_3 &= 
\begin{array}[c]{*{6}c}\cline{1-6}
\lr{1}&\lr{1}&\lr{2}&\lr{3}&\lr{4}&\lr{5}\\\cline{1-6}
\lr{2}&\lr{3}&\lr{5}&\lr{4}\\\cline{1-4}
\end{array}\\
&=
\begin{array}[c]{*{6}c}\cline{1-6}
\lr{1}&\lr{1}&\lr{2}&\lr{3}&\lr{4}&\lr{5}\\\cline{1-6}
\lr{2}&\lr{3}&\lr{4}&\lr{5}\\\cline{1-4}
\end{array}
-\begin{array}[c]{*{6}c}\cline{1-6}
\lr{1}&\lr{1}&\lr{2}&\lr{4}&\lr{4}&\lr{5}\\\cline{1-6}
\lr{2}&\lr{3}&\lr{3}&\lr{5}\\\cline{1-4}
\end{array}\\
&=t_2t_4-x_2
\end{align*}

\end{proof}
Hence the ring of invariants is generated in degree 1.

\subsection{\(\lambda=\omega_2+3\omega_1\)}
\label{sec:fv_hb2111}
In this case, the tableaux corresponding to the generators of the ring of $T$-invariants obtained from the union of Hilbert basis $H_{\SC}$ are
\[
 \begin{array}[c]{*{4}c}\cline{1-4}
 \lr{1}&\lr{3}&\lr{4}&\lr{5}\\\cline{1-4}
 \lr{2}\\\cline{1-1}
 \end{array},\hspace{20pt}
 \begin{array}[c]{*{4}c}\cline{1-4}
 \lr{1}&\lr{2}&\lr{4}&\lr{5}\\\cline{1-4}
 \lr{3}\\\cline{1-1}
 \end{array},\hspace{20pt}
 \begin{array}[c]{*{4}c}\cline{1-4}
 \lr{1}&\lr{2}&\lr{3}&\lr{5}\\\cline{1-4}
 \lr{4}\\\cline{1-1}
 \end{array},\hspace{20pt}
 \begin{array}[c]{*{4}c}\cline{1-4}
 \lr{1}&\lr{2}&\lr{3}&\lr{4}\\\cline{1-4}
 \lr{5}\\\cline{1-1}
 \end{array}
 \]


Hence the ring of invariants is generated by the first graded component.


