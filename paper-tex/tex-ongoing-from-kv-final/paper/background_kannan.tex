\subsection{Notations}
\label{sec:notations}
We first recall some well known results.   The proofs of these statements can be found in \cite{lakshmibai2007standard} and \cite{seshadriintroduction}. 
Let $e_1,\ldots,e_n$ be the standard basis of ${\mathbb C}^n$. 
Let $T$ be the maximal torus consisting of all diagonal matrices in $G$. Let $B$ be the Borel subgroup of upper triangular matrices. We denote the opposite  Borel subgroup of all lower triangular matrices in $G$ determined by $B$ and $T$ by $B^{-}$. The Weyl group of $G$ with respect to $T$ is the symmetric group $S_n$  and we have the Bruhat decomposition $G = \bigcup_w BwB$, where $w \in S_n$. Let $R$ be the set of roots for the adjoint action of $T$ on $G$. Let $S = \{\alpha_1,\ldots,\alpha_{n-1}\}$ denote the set of simple roots in $R^{+}.$ \eat{Every $\beta \in R$ can be expressed uniquely as  $\sum\limits_{i=1}^{n-1}k_{i}\alpha_{i}$ with integral coefficients $k_{i}$ all non-negative sign or non-positive. This allows us to define the {\bf height} of a root (relative to $S$) by ht$(\beta)=\sum\limits_{i=1}^{n}k_{i}.$ For $\beta=\sum\limits_{i=1}^{n}k_{i}\alpha_{i} \in R,$ we define {\bf support} of $\beta$ to be the set $\{\alpha_{i}: k_{i}\neq 0 \}.$}
The simple reflection in  $W$ corresponding to $\alpha_i$ is denoted
by $s_{i}$. Then $(W, S)$ is a Coxeter group (see \cite[Theorem 29.4, p.180]{hum1975linearalggroups}). \eat{There is a natural length function $\ell$ defined on $W.$ Let $\mathfrak{g}$ be the Lie algebra of $G$. 
Let $\mathfrak{h}\subset \mathfrak{g}$ be the Lie algebra of $T$ and  $\mathfrak{b}\subset \mathfrak{g}$ be the Lie algebra of $B$. Let $X(T)$ denote the group of all characters of $T$. 
We have $X(T)\otimes\mathbb{R}=Hom_{\mathbb{R}}(\mathfrak{h}_{\mathbb{R}}, \mathbb{R})$, the dual of the real form of $\mathfrak{h}$. The positive definite 
$W$-invariant form on $Hom_{\mathbb{R}}(\mathfrak{h}_{\mathbb{R}}, \mathbb{R})$ 
induced by the Killing form of $\mathfrak{g}$ is denoted by $(~,~)$. 
We use the notation $\left< ~,~ \right>$ to
denote $\langle \mu, \alpha \rangle  = \frac{2(\mu,
	\alpha)}{(\alpha,\alpha)}$,  for every  $\mu\in X(T)\otimes \mathbb{R}$ and $\alpha\in R$. }For a subset $J$ of $S,$ we denote by $W_{J}$ the subgroup of $W$ generated by $\{s_{\alpha}:\alpha \in J\}$. Let $W^{J}:=\{w\in W: w(\alpha)\in R^{+}~ for ~ all ~ \alpha \in J\}.$  For each $w\in W_{J},$ choose a representative element $n_{w}\in N_{G}(T).$ Let $N_{J}:=\{n_{w}: w\in W_{J}\}.$  Let $P_{J}:=BN_{J}B.$ We denote by $P^{\alpha_{r}}$, the maximal parabolic subgroup of $G$ corresponding to $S\setminus \{\alpha_r\}$.  Let  $\{\omega_{i}:1\le i\le n-1\}$ be the set of fundamental dominant weights corresponding to $\{\alpha_{i}: 1\le i\le n-1\}.$  We use the notation $W^{P_J}$ for $W^J$ and, in particular,
	$W^{S \setminus \{\alpha_r\}}$ is denoted by $W^{P^{\alpha_r}}$.
	
Let $I_{r,n}:=\{{\underline{i}} =(i_1,i_2,\ldots,i_r) |  1\leq i_1 < i_2 \cdots < i_r \leq n \}$ be the set of all strictly increasing sequences of legth $r$ with entries in $[n]$. 
 A canonical basis of $\bigwedge^r{\mathbb C}^n$ is given by $\{e_{ \underline{i}} = e_{i_1} \wedge \ldots \wedge e_{i_r}, {\underline{i}} \in I_{r,n}\}$. The Grassmannian $\grass{r}{n}$ of $r$-dimensional subspaces of ${\mathbb C}^n$ can be viewed as a subvariety of ${\mathbb P}(\bigwedge^r {\mathbb C}^n)$,
given by sending an $r$-dimensional
subspace of ${\mathbb C}^n$ with basis $v_1,v_2,\ldots,v_r$ to the class $[v_1 \wedge v_2 \wedge \cdots v_r] \in {\mathbb P}(\bigwedge^r {\mathbb C}^n)$. This identifies the Grassmannian  with the homogeneous space  $G/P^{\alpha_r}$, since  $P^{\alpha_r}$ is the  stabiliser in $G$ of the vector subspace spanned by $[e_1,e_2, \ldots,e_r]$. We denote this polarization of $G_{r,n}$ by ${\mathcal L}(\omega_r)$. When $r$ is clear from the context we also use the notation $G/P$ for the Grassmannian.

Note that the Weyl group of  $P^{\alpha_r}$ is  $W_{S\setminus \{\alpha_r\}}$. The coset representatives of $W/W_{S\setminus \{\alpha_r\}}$  of minimal length, $W^{P^{\alpha_r}}$, can be identified with $I_{r,n}$, with the coset representative $w$ corresponding to the subspace spanned by ${e_{w(1)},\ldots,e_{w(r)}}$. Note that $e_w := [e_{w(1)} \wedge e_{w(2)} \cdots \wedge e_{w(r)}]$  is a $T$ fixed point for any $w \in W^{P^{\alpha_r}}$. The $B$ orbit closure of $e_{w}$ is called a Schubert variety in $G/P$ and we denote it
by $X(w)$. $G/P$ is itself a Schubert variety for $w_0^{S \setminus \alpha_r}$, the minimal representative of the longest element $w_0$ in $W^{S \setminus \alpha_r}$.  \eat{$X(w)$ is the union of the $B$-orbits of $e_{v}$, $v \leq w$.} 

The last definition we need to recall is that of projective normality of a projective variety.
Let $X$ be a projective variety in $\mathbb{P}^{m}.$ We denote by $\hat{X}$ the affine cone of $X.$  $X$ is said to be projectively normal if $\hat{X}$ is normal. For a reference, see exercise 3.18, page. 23 of \cite{hart1977alggeo}.  

\eat{We use the following fact about projective normality of a polarized variety.
\begin{remark}
A polarized variety $(X,\mathcal{L})$ where $\mathcal{L}$ is a very-ample line bundle is said to be projectively normal if its homogeneous coordinate ring $\bigoplus\limits_{k\ge 0} H^{0}(X,\mathcal{L}^{\otimes k})$ is integrally closed and is generated as a $\mathbb{C}$-algebra by $H^0(X,\mathcal{L})$ (see exercise 5.14, chapter II of \cite{hart1977alggeo}).
\end{remark}
}
\subsection{Background}
The GIT quotient of $Gr_{r,n}$ under the action of $T$ is well studied. Gelfand and Macpherson \cite{gelfand1982geometry} studied this by considering the GIT quotient of
$n$ points spanning projective space ${\mathbb P}^{r-1}$, with respect to the group $PGL(r,\mathbb{C})$. Their results were extended by Gelfand et al in \cite{gelfand1987combinatorial}. Hausmann and Knutson\cite{hausmann1997polygon} study the GIT quotient of $Gr_{2,n}$ and related the resulting GIT quotient to the moduli space of polygons in ${\mathbb R}^3$. Skorobogatov \cite{skorobogatov1993swinnerton} gave combinatorial conditions determining when a point in $\grass{r}{n}$ is semistable with respect to the $T$-linearized bundle ${\cal L}(\omega_r)$. As a corollary he showed that when $r$ and $n$ are coprime, semistability is the same as stability. This was proved independently by  Kannan \cite{kannan1998torus}. Howard et al. \cite{howard2005projective} showed that the relations among the ring of invariants for the diagonal action of $SL(2)$ on $({\mathbb P}^1)^n$ are generated in degree four.  For $Q$ a parabolic subgroup of $G$, Howard \cite{howard2005matroids} considered the problem of determining which line bundle on $G/Q$ descends to an ample line bundle of the GIT quotient of $G/Q$ by $T$. Kumar \cite{kumar2008descent} extended these results to other algebraic groups. We use the notation ${\cal M}$ to denote the descent of the ample bundle to the GIT quotient.  Kannan and Sardar \cite{kannan2009torusA}  studied $T$-quotients of Schubert varieties in $\grass{r}{n}$ and showed that, when $r$ and $n$ are coprime there is a minimal Schubert variety in $Gr_{r,n}$ with semistable points. Kannan and Pattanayak extended these results to the case when $G$ is of type $B,C$ or $D$, and when $P$ is a maximal parabolic subgroup of $G$.  In \cite{kannan2018torus}, Kannan et al extended the results in \cite{kannan2009torusA} to Richardson varieties in the Grassmanian $\grass{r}{n}$. 
In \cite{bakshi2020torus} the authors show that $(\Lgmnmodt{2}{n}{ ({\cal L}_{n\omega_2})}, {\cal M})$ is projectively normal when $n$ is odd, using combinatorial methods. They show that the torus quotient of the  minimal Schubert variety in $Gr_{3,7}$ is projectively normal for the embedding given by ${\cal M}$. Here ${\cal M}$ is the descent of ${\cal L}(7 \omega_3)$.  They leave open the question of whether $(\Lgmnmodt{3}{7}{ ({\cal L}_{\omega_3})}, {\cal M})$ is projectively normal.


 \subsection{Main results}
We give a computational proof  that the polarised variety ($\Lgmnmodt{3}{7}{({\cal L}_{7\omega_3})}, {\cal M})$ is projectively normal.  
We also study $T\backslash\mkern-6mu\backslash(SL(5, \mathbb{C})/Q)^{ss}_{T} {\mathcal L}(\lambda)$. Here $Q=P^{\alpha_{r}}\cap P^{\alpha_{s}}$ for some integers $1\leq r < s \leq 4$ and $\lambda=a\omega_r+b\omega_s$, with $a$ and $b$ positive integers such that $ar+bs=5$. 
We give complete descriptions of the GIT quotients in these cases.

\subsection{Organisation}
In Section \ref{sec:std:monom} we recall relevant background on standard monomials that we use.
In Section~\ref{sec:hb_grass} we describe the subring of $T$-invariants. We then recall definitions from polyhedral theory. This allows us to set up a bijection between $T$-invariant monomials in the coordinate ring of $Gr_{r,n}$ and unions of lattice points of pointed cones. In Section \ref{sec:g37} we use this connection to give a computational proof that $(\gmnmodt{3}{7}{\cal L}(7\omega_3), {\cal M})$  is projectively normal. In Section~\ref{sec:fv_normality} we give complete descriptions of  $T \backslash\mkern-6mu\backslash (SL(5,{\mathbb C}) / Q)^{ss}_{T}{\cal L}(\lambda)$ when $Q$ is the intersection of two maximal parabolic subgroups of $SL(5,{\mathbb C})$, and $\lambda$ is as described in the above paragraph.

%\bibitem{Har} Robin Hartshorne, {\em Algebraic Geometry,  Graduate Texts in Mathematics book series }(GTM, volume 52).

%\bibitem{Kum} Shrawan Kumar, {\em Descent of line bundles to GIT quotients of flag varieties by maximal torus,} Transformation groups, vol.13, no. 3-4, p. 757--771,2008.



