\section{Generating set of $T$-invariants using polyhedral theory}
\label{hb_as_Tinv_gens}
Since $H^0(Gr_{r,n}, {\cal L}(kn\omega_r))$ is spanned by standard monomials, the homogeneous coordinate ring of the $T$-quotient, $R=\oplus_k R(k)$, is spanned by $T$-invariants which are standard. We call these standard $T$-invariants.  Now, standardness is defined by the "$\leq$" relation on the Br\"{u}hat poset. A $T$-invariant monomial which is the product of Pl\"{u}cker coordinates with support in a maximal chain in the Br\"{u}hat poset is naturally standard. 

This gives us a natural recipe to find generators for the ring of invariants. In the first step, for every maximal chain ${\cal C}$ in the Br\"{u}hat poset find a generating set of $T$-invariant monomials which can be expressed as a product of Pl\"{u}cker coordinates with support in ${\cal C}$. Then take the union of these generating sets of $T$-invariant monomials over all maximal chains in the Br\"{u}hat poset. This follows, since every standard monomial $p_{S}$which is $T$-invariant has Pl\"{u}cker coordinates which are pairwise comparable, and so there is a maximal chain in the Br\"{u}hat poset which contains the support of $p_{S}$.  This is the approach we take. In the final step we prune the generating set using the straightening relations. To carry out the first step, we use polyhedral theory. We describe that in the next section.  
It will be clear that with a little work this approach can be extended to partial flag varieties also.  While this approach is general, it is computationally prohibitive. We are able to carry this program out for $Gr_{3,7}$ and flag varieties of of the form $SL(5,{\mathbb C})/Q$, where $Q$ is the intersection of two maximal parabolic subgroups of $SL(5,{\mathbb C})$ containing $B$.   

\subsection{$T$ invariants with support in a maximal chain}
\label{sec:formulation}
We formulate the question of obtaining a generating set of $T$-invariant monomials with support in a fixed maximal chain ${\mathcal C}$, of the Bruhat poset using notions from polyhedral theory.  We recall the relevant definitions. A standard and wonderful reference for all this is Schrijver's book \cite{schrijver1998theory}. 

\begin{definition}[Cone]
    A  subset \(C \subseteq  \IR^n\) is a cone if for any \(x,y\in C\) and non-negative real numbers \(\lambda_1,\lambda_2\), we have \(\lambda_1 x + \lambda_2 y \in P\). A cone \(C\) is pointed if \(C \cap -C = \{0\}\). 
    A cone \(C\) is polyhedral if \(C = \{ x \in \IR^n | Mx \ge 0\}\) for some $m \times n$ real matrix \(M\). 
    A set $\{g_1,...g_k\} \subset C$ is called a (conical) generating set  of \(C\) if for all \(x \in C\) there exist non-negative real numbers \(\lambda_1,...,\lambda_k\) such that \(x = \sum_{i=1}^k \lambda_i g_i\).
\end{definition}

\begin{definition}[Hilbert basis]
    Let \(C \in \IR^n\) be a polyhedral cone with rational generators. \eat{and 
    let \(\Lambda \subset \IZ^n\) be a lattice.}
    We call a finite set \(H = \{h_1,...,h_t\} \subset \IZ^n \cap C\) a Hilbert basis of $C$,
   if for every integral vector $v \in C$ there exist non-negative integers \(\lambda_1,...,\lambda_t\) such that 
    $v=\sum_{i=1}^t \lambda_i h_i.$ 
  \end{definition} 
  
  The following is well known, see \cite[Theorem 16.4]{schrijver1998theory}.   
 \begin{proposition}
Every rational polyhedral pointed cone has a unique inclusion minimal Hilbert basis.
 \end{proposition}
 
We will show that $T$-invariant monomials in a fixed maximal chain $\SC$ in the Br\"{u}hat order are in bijection with integer points in a polyhedral cone $C_{\SC}$. It will then follow that monomials corresponding to the Hilbert basis of integer points in $C_{\SC}$ will be a generating set for invariants whose support is in $\SC$, and these will be standard by the discussion in the previous section.

Let \(\SC = {\UL{i}_1} \le \cdots \le  {\UL{i}_l}\) be a \eat{maximal }chain in the Bruhat poset \(I_{r,n}\). For every standard monomial \(p_{\UL{i}_1}^{a_{\UL{i}_1}}p_{\UL{i}_2}^{a_{\UL{i}_2}}...p_{\UL{i}_l}^{a_{\UL{i}_l}}\) of degree \(k\), we have a semistandard tableau \(S\) of rectangular shape \((k^r)\) such that column \(\UL{i}_j\) appears \(a_{\UL{i}_j}\) times in \(S\), for  each \(1 \le j \le l\). Note that some of \(a_{{\UL i}_j}\) may be \(0\). 
Associate vectors \(v'_{_S} = (a_{\UL{i}_1},...,a_{\UL{i}_l})\) and \(v_{_S} = (v'_{_S},\frac{k}{n})\) with this standard monomial.
If \(X,Y\) are tableaux with support in \(\SC\)  and \(XY\) is the tableau associated with monomial \(p_{_X}p_{_Y}\) then clearly \(v'_{_{XY}} = v'_{_X} + v'_{_Y}\) and \(v_{_{XY}} = v_{_X}+v_{_Y}\).

    We associate with $\SC$ an \(n \times l\) matrix \(A'_{\SC}\) with rows indexed by \([n]\) and columns indexed by Pl\"{u}cker coordinates with support in the chain \(\SC\) as follows: for \(i \in [n],\UL{j} \in \SC\),
    \[
        (A'_{\SC})_{i,\UL{j}}=
        \left\{\begin{array}{cc}
            1  &  i \in \UL{j}\\
            0  &  i \not\in \UL{j}
        \end{array}\right.
    \]
That is the \(\UL{j}^{th}\) column of \(A'_{\SC}\) is weight vector of column tableau \(\UL{j}\). Let \(v = (-r,\cdots,-r)\) be a \(n\)-dimentional vector. Let \(A_{\SC} \DEF [A'_{\SC}|v]\) be the matrix \(A'_{\SC}\) augmented with column \(v\). Let \(C_{\SC} = \IR_{\ge0}^{l+1} \cap ker(A_{\SC})\). We have following observations.
    \begin{observation}
        \label{obs:tableauasvector}
        Let \(S\) be a tableau whose columns are in \(\SC\). For \(i \in [n]\) we have
        \begin{center}
        \((A'_{\SC}v'_{_S})_{i} = \) number of columns in \(S\) which contain \(i\).
        \end{center}
        Hence, \(wt(S) = A'_{\SC}v'_{_S}\). If \(S\) is \(T\)-invariant then \(v_{_S}\) is a nonzero integral vector in \(C_{\SC}\).
    \end{observation}
    \begin{proof}
        \[
            (A'_{\SC}v'_{_S})_{i} 
            = \sum_{\UL{j} \in \SC} (A'_{\SC})_{i,\UL{j}} (v'_{_S})_{\UL{j}}
            = \sum_{i \in \UL{j},\UL{j}\in \SC} (v'_{_S})_{\UL{j}}.
        \]
        This proves \(wt(S) = A'_{\SC}v'_{\SC}\). 

        To prove that \(v_{_S}\) is integral vector for a \(T\)-invariant tableau \(S\), we need to show that \(\frac{k}{n}\) is integer. Observe that since \(wt(S) = (\frac{kr}{n},\cdots,\frac{kr}{n})\) is an integer vector we have \(n|k\) since \((r,n)=1\).  \(v_{_S} \in ker(A_{\SC})\) follows from definition of \(A_{\SC}\). The non-negativity of \(v_{_S}\) follows from definition of \(v_{_S}\).
    \end{proof}
    \begin{observation}
        Let \(x=(x',d)\in \IZ_{\ge0}^{l+1}\cap C_{\SC}\) be a nonzero integer vector and suppose $d=\frac{k}{n}$ for some \(k\). Then there exists a \(T\)-invariant semistandard tableau \(S\) of degree \(k\) such that \(x=v_{_S}\).
    \end{observation}
    \begin{proof}
        Define the tableau \(S\) with \(\UL{j} \in \SC\) appearing in \(S\),  \(x'_{\UL{j}}\) times.
        Then we have \(x' = v'_{_S}\).
        Since the columns of \(S\) have support in $\SC$, these columns can be sorted so that \(S\) is semistandard.
        It remains to show that \(S\) is \(T\)-invariant and of degree \(k\). 
        That is \(wt(S) = (\frac{kr}{n},\cdots,\frac{kr}{n})\).
        This follows from calculation below.
        \[
            A_{\SC}x = 0
            \implies [A'_{\SC}|w](x',\frac{k}{n}) = 0
            \implies A'_{\SC}x' = -\frac{k}{n}w
            \implies wt(S) = (\frac{kr}{n},\cdots,\frac{kr}{n}).
        \]
    \end{proof}

\begin{corollary}
\label{cor:inv-cone}
We have a bijection between \(T\) invariant monomials with support in \(\SC\) and nonzero integral points in \(C_{\SC}\) given by tableau \(S \mapsto v_{_S}\). 
\end{corollary}
\begin{proof}
   This follows from the above observations.
\end{proof}
\begin{corollary}
\label{cor:Hilb}
$C_{\SC} = ker(A_{\SC}) \cap {\mathbb R}_{\ge0}^{l+1}$ is a pointed cone. Hence $C_{\SC}$ has a unique inclusion minimal Hilbert basis.
\end{corollary}
\begin{proof}
   This follows from the observation that any subspace intersected with a pointed cone (the non-negative orthant in our case) results in a pointed cone.
\end{proof}

\subsection{Generating set of $T$-invariants}
For a maximal chain $\SC$, let $H_{\SC}$ denote the unique Hilbert basis of the pointed polyhedral cone  \(C_{\SC}\).
\eat{Let \(H = \bigcup H_{\SC}\) where union is taken over all maximal chains in \(I_{r,n}\).}

\begin{definition}
    We say a \(T\) invariant semistandard tableau \(S\) splits directly 
    if there are two \(T\) invariant semistandard tableaux \(X,Y\) such that \(S=XY\) up to rearrangement of columns. 
\end{definition}

The following lemma is now immediate.
\begin{lemma}
    \label{lemma:grass_directsplittinglemma}
    Let \(S\) be any \(T\) invariant semistandard tableau. \(S\) splits directly iff \(v_{_S} \not\in H_{\SC}\ \) for any maximal chain $\SC$ in \(I_{r,n}\).
\end{lemma}
\begin{proof} 
    Let \(\SC\) be any maximal chain such that all columns of \(S\) are in \(\SC\).

    \((\implies)\)
    Let \(S\) splits directly then there are \(T\)-invariant semistandard tableaux \(X,Y\) such that 
    \(S = XY\) up to rearranging columns. 
    This implies \(v_{_S} = v_{_X} + v_{_Y}\) in the chain \(\SC\) 
    and thus \(v_{_S}\) is decomposible hence \(v_{_S} \not\in H_{\SC}\).

    \((\impliedby)\) If \(v_{_S} \not\in H_{\SC}\) then there are non-negative integers 
    \(\lambda_1,...,\lambda_t\) such that \(v_{_S} = \sum_{i=1}^t \lambda_i h_i\) and \(\sum_{i=1}^t \lambda_i \ge 2\) (otherwise \(v_{_S} \in H\)). 
    This implies there is \(u\in \IZ^{l+1}\cap C_{\SC} \) and \(h,h' \in H\) 
    such that \(v_{_S} = h + h' + u\). Hence \(S = XYZ\) up to rearranging columns where \(X,Y,Z\) are \(T\)-invariant tableaux such that \(v_{_X}=h,v_{_Y}=h',v_{_Z}=u\).   So $S$ splits, completing the proof.
\end{proof}

We have,
\begin{theorem}
\label{thm:main}
    The coordinate ring of \(\Lgmnmodt{r}{n}{{\cal L}(n\omega_r)}\) is generated by the union of standard monomials corresponding to elements in the Hilbert basis $H_{\SC}$, the union taken over all maximal chains $\SC$ in \(I_{r,n}\).
\end{theorem}
\begin{proof}
The proof follows from the definition of a Hilbert basis of a polyhedral cone, the definition of a split tableau and Corollary~\ref{cor:inv-cone}, Corollary~\ref{cor:Hilb} and Lemma~\ref{lemma:grass_directsplittinglemma}.

\end{proof}

In fact the above proof works for any Schubert variety in the Grassmannian. We state this for completeness.
\begin{corollary}
\label{cor:schub}
    Let \(w \in I_{r,n}\) and \({\cal P}\subset I_{r,n}\) be the sub-poset consisting of elements $v \in I_{r,n}$ such that $v \leq w$. The coordinate ring of \(\LXWmodT{w}{{\cal L}(n\omega_r)}\) is generated by the union of standard monomials corresponding to elements in the Hilbert basis $H_{\SC}$, where the union is taken over all maximal chains $\SC$ in the poset \({\cal P}\).
\end{corollary}
\begin{proof}
 Clear.
\end{proof}

