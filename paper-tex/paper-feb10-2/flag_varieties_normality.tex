\section{$T$-quotients of partial flag varities}
\label{sec:fv_normality}
We use ideas similar to what we used in the previous section to study GIT quotients $T\backslash\mkern-6mu\backslash(SL(5, \mathbb{C}) \allowbreak /Q)^{ss}_{T} {\mathcal L}(\lambda)$, for certain parabolic subgroups $Q$. 

\begin{theorem}
\label{thm:flag}
    We have the following:
\begin{enumerate}
    \item Let $Q = P^{\alpha_3} \cap P^{\alpha_2}$ and $\lambda = \omega_3 + \omega_2$. Let $X = \flvmodt {\cal L}(\lambda)$ and \(\cal M\) be the descent of \(\cal L(\lambda)\) to \(X\).  The homogeneous coordinate ring of $(X,{\cal M})$ is an integral extension of a polynomial ring in five variables.
    \item Let $Q = P^{\alpha_4} \cap P^{\alpha_1}$ and $\lambda = \omega_4 + \omega_1$. Let $X = \flvmodt {\cal L}(\lambda)$ and \(\cal M\) be the descent of \({\cal L}(\lambda)\) to \(X\). The polarized variety $(X,{\cal M})$ is projectively normal.
    \item Let $Q = P^{\alpha_3} \cap P^{\alpha_1}$ and $\lambda = \omega_3 + 2\omega_1$. Let $X = \flvmodt {\cal L}(\lambda)$ and \(\cal M\) be the descent of \({\cal L}(\lambda)\) to \(X\). The polarized variety $(X,{\cal M})$ is a projectively normal.
    \item Let $Q= P^{\alpha_2} \cap P^{\alpha_1}$ and $\lambda_1 = 2\omega_2 + \omega_1$ and $\lambda_2 = \omega_2 + 3 \omega_1$. For $i=1,2$ let $X_i = \flvmodt {\cal L}(\lambda_i)$ and \({\cal M}_i\) be the descent of \({\cal L}(\lambda_i)\) to \(X_i\).  The polarized variety $(X_i,{\cal M}_i)$ is projectively normal.
\end{enumerate}
\end{theorem}

\begin{proof}
    In all cases we follow the procedure outlined in Section~\ref{hb_as_Tinv_gens}. For each $Q$ we set up an appropriate poset ${\cal P}$. Given $\lambda$, for each maximal chain $\SC$ in ${\cal P}$, we construct a pointed cone $C_{\SC, \lambda}$. We compute a Hilbert basis of $C_{\SC, \lambda}$.  The standard monomials corresponding to the Hilbert basis elements have support in $\SC$, and the shape of the tableau is a non-negative integral multiple of the shape of $\lambda$. The union of standard monomials corresponding to these Hilbert basis gives us a generating set for the ring of $T$-invariants of $SL(5, \mathbb{C})/Q$, with respect to the polarization $\lambda$. 
    
For each $Q$ and $\lambda$ as in the statement of the theorem, details of the construction of ${\cal P}$ and details of the construction of  $C_{\SC, \lambda}$ for each maximal chain $\SC$ are given in Appendix~\ref{sec:fv_notations}\footnote{To keep the notation simple, we omit the additional subscript $\lambda$ when referring to $C_{\SC, \lambda}$ in the appendix.}.
    
For each $Q$, $\lambda$, in Appendix~\ref{sec:fv_hb} we write down explicitly the generators of the ring of $T$-invariants obtained. The theorem follows from the calculations given there. Here, we only present details of part 1 of the theorem, and illustrate via an example the construction of $C_{\SC, \lambda}$. Again, for simplicity of notation we omit the subscript $\lambda$ in  $C_{\SC, \lambda}$.

 Let \(Q = P^{\alpha_3} \cap P^{\alpha_2}\) and \(\lambda = \omega_3 + \omega_2\). The poset in this case is on set \(I_{Q} \DEF I_{3,5}\cup I_{2,5}\).
    We extend the Br\"{u}hat order on \(I_{2,5}\) and \(I_{3,5}\) to \(I_{Q}\) as follows:
\begin{center}
    For ${\UL i}=[i_1,i_2,i_3]$ and ${\UL j}=[j_1,j_2]$ define ${\UL i} \leq {\UL j}$ if $i_1 \leq j_1$ and $i_2 \leq j_2$.
\end{center}
    The poset \(I_{Q}\) contains 20 elements and 42 maximal chains.
    We fix the following maximal chain of length \(l=10\).
    
 \[
\SC = 
\begin{array}[c]{*{1}c}\cline{1-1}
\lr{1}\\\cline{1-1}
\lr{2}\\\cline{1-1}
\lr{3}\\\cline{1-1}
\end{array}
\le
\begin{array}[c]{*{1}c}\cline{1-1}
\lr{1}\\\cline{1-1}
\lr{2}\\\cline{1-1}
\lr{4}\\\cline{1-1}
\end{array}
\le
\begin{array}[c]{*{1}c}\cline{1-1}
\lr{1}\\\cline{1-1}
\lr{2}\\\cline{1-1}
\lr{5}\\\cline{1-1}
\end{array}
\le
\begin{array}[c]{*{1}c}\cline{1-1}
\lr{1}\\\cline{1-1}
\lr{2}\\\cline{1-1}
\end{array}
\le
\begin{array}[c]{*{1}c}\cline{1-1}
\lr{1}\\\cline{1-1}
\lr{3}\\\cline{1-1}
\end{array}
\le
\begin{array}[c]{*{1}c}\cline{1-1}
\lr{1}\\\cline{1-1}
\lr{4}\\\cline{1-1}
\end{array}
\le
\begin{array}[c]{*{1}c}\cline{1-1}
\lr{1}\\\cline{1-1}
\lr{5}\\\cline{1-1}
\end{array}
\le
\begin{array}[c]{*{1}c}\cline{1-1}
\lr{2}\\\cline{1-1}
\lr{5}\\\cline{1-1}
\end{array}
\le
\begin{array}[c]{*{1}c}\cline{1-1}
\lr{3}\\\cline{1-1}
\lr{5}\\\cline{1-1}
\end{array}
\le
\begin{array}[c]{*{1}c}\cline{1-1}
\lr{4}\\\cline{1-1}
\lr{5}\\\cline{1-1}
\end{array}
\]
\unskip 
 Following the construction given in Appendix~\ref{sec:fv_notations}, the matrices \(A_{\SC}\) and \(B_{\SC}\) are as follows.
\[
A_{\SC} = 
 \left(\begin{array}{rrrrrrrrrrr}
1 & 1 & 1 & 1 & 1 & 1 & 1 & 0 & 0 & 0 & -1 \\
1 & 1 & 1 & 1 & 0 & 0 & 0 & 1 & 0 & 0 & -1 \\
1 & 0 & 0 & 0 & 1 & 0 & 0 & 0 & 1 & 0 & -1 \\
0 & 1 & 0 & 0 & 0 & 1 & 0 & 0 & 0 & 1 & -1 \\
0 & 0 & 1 & 0 & 0 & 0 & 1 & 1 & 1 & 1 & -1 
\end{array}\right) 
\]

\[
B_{\SC} = 
 \left(\begin{array}{rrrrrrrrrrr}
1 & 1 & 1 & 0 & 0 & 0 & 0 & 0 & 0 & 0 & -1 \\
0 & 0 & 0 & 1 & 1 & 1 & 1 & 1 & 1 & 1 & -1
\end{array}\right) 
\]
\unskip
    The matrix \(A_{\SC}\) is of dimension \(5 \times 11 \) and the matrix \(B_{\SC}\) has dimension \(2 \times 11\). 
Set \(P_{\SC} = ker(A_{\SC}) \cap ker(B_{\SC}) \cap \IR_{\ge 0}^{l+1}\). 
    We get 
\[
H_{\SC}=
\left\{\begin{array}{c}
(0, 1, 0, 0, 0, 0, 0, 0, 0, 0, 0, 0, 0, 0, 0, 0, 0, 0, 1, 0, 1)\\
(1, 0, 0, 0, 0, 0, 0, 0, 0, 0, 0, 0, 0, 0, 0, 0, 0, 0, 0, 1, 1)
\end{array}\right\}
\]
and the two tableaux corresponding to above vectors are:
\[
t_{0}=\begin{array}[c]{*{2}c}\cline{1-2}
\lr{1}&\lr{3}\\\cline{1-2}
\lr{2}&\lr{5}\\\cline{1-2}
\lr{4}\\\cline{1-1}
\end{array}\hspace{20px}
t_{1}=\begin{array}[c]{*{2}c}\cline{1-2}
\lr{1}&\lr{4}\\\cline{1-2}
\lr{2}&\lr{5}\\\cline{1-2}
\lr{3}\\\cline{1-1}
\end{array}
\]
\unskip
    Similarly, we compute the Hilbert basis for all maximal chains in \(I_{Q}\) and we get following generating set for ring of \(T\)-invariants. (Observe that tableaux \(t_0\) and \(t_1\) below have support in the chain \(\SC\) above.)  
t_{0}&=\begin{array}[c]{*{2}c}\cline{1-2}
\lr{1}&\lr{3}\\\cline{1-2}
\lr{2}&\lr{5}\\\cline{1-2}
\lr{4}\\\cline{1-1}
\end{array}
\hspace{20px}t_{1}&=\begin{array}[c]{*{2}c}\cline{1-2}
\lr{1}&\lr{4}\\\cline{1-2}
\lr{2}&\lr{5}\\\cline{1-2}
\lr{3}\\\cline{1-1}
\end{array}
\hspace{20px}t_{2}&=\begin{array}[c]{*{2}c}\cline{1-2}
\lr{1}&\lr{3}\\\cline{1-2}
\lr{2}&\lr{4}\\\cline{1-2}
\lr{5}\\\cline{1-1}
\end{array}
\hspace{20px}t_{3}&=\begin{array}[c]{*{2}c}\cline{1-2}
\lr{1}&\lr{2}\\\cline{1-2}
\lr{3}&\lr{4}\\\cline{1-2}
\lr{5}\\\cline{1-1}
\end{array}
\hspace{20px}t_{4}&=\begin{array}[c]{*{4}c}\cline{1-4}
\lr{1}&\lr{1}&\lr{2}&\lr{4}\\\cline{1-4}
\lr{2}&\lr{3}&\lr{3}&\lr{5}\\\cline{1-4}
\lr{4}&\lr{5}\\\cline{1-2}
\end{array}
\hspace{20px}t_{5}&=\begin{array}[c]{*{4}c}\cline{1-4}
\lr{1}&\lr{1}&\lr{2}&\lr{3}\\\cline{1-4}
\lr{2}&\lr{4}&\lr{4}&\lr{5}\\\cline{1-4}
\lr{3}&\lr{5}\\\cline{1-2}
\end{array}
\hspace{20px}t_{6}&=\begin{array}[c]{*{2}c}\cline{1-2}
\lr{1}&\lr{2}\\\cline{1-2}
\lr{3}&\lr{5}\\\cline{1-2}
\lr{4}\\\cline{1-1}
\end{array}
\hspace{20px}
\unskip
    From Theorem~\ref{thm:main}, the ring of invariants is generated by \(t_0,...,t_4,x_1,x_2\), hence it is contained in \(\IC[t_0,\cdots,t_4,x_1,x_2]\). 
    To prove part 1 of the claim we further study the ring of invariants. Observe that only \(t_2t_3\) and \(x_1x_2\) are the two nonstandard products amongst \(t_0,\cdots,t_4,x_1,x_2\).
    By straightening \(t_2t_4\) and \(x_1x_2\), it can be easily seen that
    \begin{align*}
        x_1x_2 &= f_1(t_0,t_1,t_2,t_3,t_4),\\ 
        x_1+x_2 &= f_2(t_0,t_1,t_2,t_3,t_4).
    \end{align*}
    Here $f_1$ and $f_2$ are in the ring \(\IC[t_0,...,t_4]\).  Note that the ring \(\IC[t_0,...,t_4]\) is a polynomial ring in five variables. It also follows from the above calculations that $x_1-x_2$ satisfies a monic polynomial of degree 2 with coefficients in the ring \(\IC[t_0,...,t_4]\). So $x_1, x_2$ are integral over \(\IC[t_0,\cdots,t_4]\) and hence \(\IC[t_0,...,t_4,x_1,x_2]\) is an integral extension of \(\IC[t_0,...,t_4]\). This completes the proof of the theorem.
We give details of the polynomials $f_1$ and $f_2$ in Appendix~\ref{sec:fv_hb40}. Proofs for the other parts are similar and are presented later in Appendix~\ref{sec:fv_hb}.

\end{proof}
