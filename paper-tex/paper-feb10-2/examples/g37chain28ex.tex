
\subsubsection{Example}

For \(\grass{3}{7}\), the Br\"{u}hat poset is on the set \(I(3,7)\) which has cardinality 35. There are 462 maximal chains in the Br\"{u}hat poset with lowest element \([1,2,3]\) and top element \([5,6,7]\). The length of every maximal chain is \(13\). Let \(\SC\) be the following maximal chain in \(I_{3,7}\). 

\[
\SC = 
\begin{array}[c]{*{1}c}\cline{1-1}
\lr{1}\\\cline{1-1}
\lr{2}\\\cline{1-1}
\lr{3}\\\cline{1-1}
\end{array}
\le
\begin{array}[c]{*{1}c}\cline{1-1}
\lr{1}\\\cline{1-1}
\lr{2}\\\cline{1-1}
\lr{4}\\\cline{1-1}
\end{array}
\le
\begin{array}[c]{*{1}c}\cline{1-1}
\lr{1}\\\cline{1-1}
\lr{2}\\\cline{1-1}
\lr{5}\\\cline{1-1}
\end{array}
\le
\begin{array}[c]{*{1}c}\cline{1-1}
\lr{1}\\\cline{1-1}
\lr{2}\\\cline{1-1}
\lr{6}\\\cline{1-1}
\end{array}
\le
\begin{array}[c]{*{1}c}\cline{1-1}
\lr{1}\\\cline{1-1}
\lr{3}\\\cline{1-1}
\lr{6}\\\cline{1-1}
\end{array}
\le
\begin{array}[c]{*{1}c}\cline{1-1}
\lr{2}\\\cline{1-1}
\lr{3}\\\cline{1-1}
\lr{6}\\\cline{1-1}
\end{array}
\le
\begin{array}[c]{*{1}c}\cline{1-1}
\lr{2}\\\cline{1-1}
\lr{3}\\\cline{1-1}
\lr{7}\\\cline{1-1}
\end{array}
\le
\begin{array}[c]{*{1}c}\cline{1-1}
\lr{2}\\\cline{1-1}
\lr{4}\\\cline{1-1}
\lr{7}\\\cline{1-1}
\end{array}
\le
\begin{array}[c]{*{1}c}\cline{1-1}
\lr{3}\\\cline{1-1}
\lr{4}\\\cline{1-1}
\lr{7}\\\cline{1-1}
\end{array}
\le
\begin{array}[c]{*{1}c}\cline{1-1}
\lr{3}\\\cline{1-1}
\lr{5}\\\cline{1-1}
\lr{7}\\\cline{1-1}
\end{array}
\le
\begin{array}[c]{*{1}c}\cline{1-1}
\lr{4}\\\cline{1-1}
\lr{5}\\\cline{1-1}
\lr{7}\\\cline{1-1}
\end{array}
\le
\begin{array}[c]{*{1}c}\cline{1-1}
\lr{4}\\\cline{1-1}
\lr{6}\\\cline{1-1}
\lr{7}\\\cline{1-1}
\end{array}
\le
\begin{array}[c]{*{1}c}\cline{1-1}
\lr{5}\\\cline{1-1}
\lr{6}\\\cline{1-1}
\lr{7}\\\cline{1-1}
\end{array}
\]
The matrix \(A'_{\SC}\) is of order \(7 \times 13\) and \(v= (-3,\cdots,-3)\). Hence we get the following matrix \(A_{\SC}=[A'_{\SC}|v]\) of order \(7 \times 14\).
\[
A_{\SC} = 
\left(\begin{array}{rrrrrrrrrrrrrr}
    1 & 1 & 1 & 1 & 1 & 0 & 0 & 0 & 0 & 0 & 0 & 0 & 0 & -3 \\
    1 & 1 & 1 & 1 & 0 & 1 & 1 & 1 & 0 & 0 & 0 & 0 & 0 & -3 \\
    1 & 0 & 0 & 0 & 1 & 1 & 1 & 0 & 1 & 1 & 0 & 0 & 0 & -3 \\
    0 & 1 & 0 & 0 & 0 & 0 & 0 & 1 & 1 & 0 & 1 & 1 & 0 & -3 \\
    0 & 0 & 1 & 0 & 0 & 0 & 0 & 0 & 0 & 1 & 1 & 0 & 1 & -3 \\
    0 & 0 & 0 & 1 & 1 & 1 & 0 & 0 & 0 & 0 & 0 & 1 & 1 & -3 \\
    0 & 0 & 0 & 0 & 0 & 0 & 1 & 1 & 1 & 1 & 1 & 1 & 1 & -3
\end{array}\right)
\]
The monomial indexed by the following tableau is an example of a $T$-invariant polynomial with support in \(\SC\).
\[S = 
    \begin{array}[c]{*{14}c}\cline{1-14}
    \lr{1}&\lr{1}&\lr{1}&\lr{1}&\lr{1}&\lr{1}&\lr{2}&\lr{2}&\lr{2}&\lr{4}&\lr{4}&\lr{4}&\lr{4}&\lr{4}\\\cline{1-14}
    \lr{2}&\lr{2}&\lr{2}&\lr{3}&\lr{3}&\lr{3}&\lr{3}&\lr{3}&\lr{3}&\lr{5}&\lr{5}&\lr{5}&\lr{5}&\lr{5}\\\cline{1-14}
    \lr{4}&\lr{5}&\lr{6}&\lr{6}&\lr{6}&\lr{6}&\lr{6}&\lr{6}&\lr{7}&\lr{7}&\lr{7}&\lr{7}&\lr{7}&\lr{7}\\\cline{1-14}
    \end{array}
    \hspace{4pt}, \hspace{5pt}
    \VEC{v}_{_{S}} = [0, 1, 1, 1, 3, 2, 1, 0, 0, 0, 5, 0, 0, 2].
\]
Observe that \(S\) is degree 14 monomial hence \(\frac{k}{n} = 2 \) which is the last entry in \(v_{_S}\).  
