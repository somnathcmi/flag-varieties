\newcommand\hilbert{

\subsection{Hilbert basis}
\npara
\begin{definition}[Cone]
    A set \(P\in \IR^n\) is cone if for any \(x,y\in P\) and non-negative \(\lambda_1,\lambda_2\),
    we have \(\lambda_1 x + \lambda_2 y \in P\). A cone \(P\) is pointed if \(P \cap -P = \{0\}\). 
    A cone \(P\) is polyhedral if \(P = \{ x \in \IR^n | Bx \ge 0\}\) for some matrix \(B\). 
    Set \(\{g_1,...g_k\}\) is called generators of \(P\) 
    if for all \(x \in P\) there are non-negative \(\lambda_1,...,\lambda_k\) 
    such that \(x = \sum_{i=1}^k \lambda_i c_i\).
\end{definition}
\begin{definition}[Lattice]
    A lattice in \(\IR^n\) is subgroup of \(\IR^n\) under addition.
\end{definition}
\begin{definition}[From paper on Hilbert basis]
    Let \(P \in \IR^n\) be a polyhedral cone with rational generators and 
    let \(\Lambda \subset \IZ^n\) be a lattice. 
    We call finite set \(H = \{h_1,...,h_t\} \subset \Lambda \cap P\) generate monoid 
    \((\Lambda \cap P,+)\) if for every \(x \in \Lambda \cap P\) 
    there are non-negative integers \(\lambda_1,...,\lambda_t\) such that 
    \(\sum_{i=1}^t \lambda_i h_i\). 
    Following are known results:
    \begin{enumerate}
        \item
            If \(P\) is pointed cone then there is unique inclusion minimal generating set.
        \item
            If \(\Lambda = \IZ^n\) and \(P\) is pointed cone then the inclusion minimal set \(H\) 
            is called Hilbert basis of monoid \(P\).
        \item 
            Every element of hilbert basis is indecomposible, i.e., 
            it cannot be writen as sum of two elements of the monoid.
    \end{enumerate}
\end{definition}

Note that \(P_{\SC} = ker(A)_{_{\ge 0}}\) is pointed cone:
1. Addition of two non-negative vectors in \(ker(A)\) is non-negative and in \(ker(A)\). 
2. Non-negative scalar multiple of non-negative vector in \(ker(A)\) 
is non-negative and in \(ker(A)\). 
3. \(x,-x \in P_{\SC} \implies x = 0\). 
\begin{definition}
    We say \(T \in STab(r,n,\SC)\) splits directly 
    if there are two tableaux \(X,Y \in STab(r,n,\SC)\) such that \(T=XY\) up to rearranging columns.
\end{definition}

\begin{lemma}
    \(T \in STab(r,n,\SC)\) splits directly iff \(v_{_T} \not\in H\). 
    Where \(H\) is hilbert basis of \(P_{\SC}\).
\end{lemma}
\begin{proof}
    \((\implies)\)
    Let \(T\) splits directly then there exist \(X,Y \in STab(r,n,\SC)\) such that 
    \(T = XY\) up to rearranging columns. This implies \(v_{_T} = v_{_X} + v_{_Y}\) 
    and thus \(v_{T}\) is decomposible hence \(v_{_T} \not\in H\).

    \((\impliedby)\) If \(v_{_T} \not\in H\) then there are non-negative integers 
    \(\lambda_1,...,\lambda_t\) such that \(v_{_T} = \sum_{i=1}^t \lambda_i h_i\). 
    Note that \(\sum_{i=1}^t \lambda_i \ge 2\), if not then \(v_{_T} \in H\). 
    This implies there is \(u\in Z^{l+1}\cap P_{\SC} \) and \(h,h' \in H\) 
    such that \(v_{_T} = h + h' + u\). 
    Using property \(v_{_{ST}} = v_{_S} + v_{_T}\) and corollary 4, we get \(T=T^{h}T^{h'}T^u\).     
\end{proof}
%END
}
