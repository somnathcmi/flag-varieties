%
\documentclass[11pt]{article}
\usepackage{amsmath,amssymb,amsthm}
\usepackage{graphicx}
\usepackage{datetime}
\usepackage[margin=1in]{geometry}
\usepackage{fancyhdr}
\usepackage{mathtools}
\usepackage{amsmath}
\usepackage{bm}
\usepackage[mathscr]{euscript}
\usepackage{xcolor}
\usepackage{algorithm2e}
\usepackage{enumitem}
\usepackage{stmaryrd}
\renewcommand{\familydefault}{\rmdefault}


%to do: add all possible mathsymbols reference website http://www.peteryu.ca/tutorials/publishing/latex_math_script_styles
\newcommand\MFb{\mathfrak{b}}
\newcommand\MFt{\mathfrak{t}}
\newcommand\MFw{\mathfrak{w}}
\newcommand\IR{\mathbb{R}}
\newcommand\IN{\mathbb{N}}
\newcommand\IZ{\mathbb{Z}}
\newcommand\IC{\mathbb{C}}
\newcommand\IQ{\mathbb{Q}}
\newcommand\IF{\mathbb{F}}
\newcommand\SF{\mathscr{F}}
\newcommand\SC{\mathscr{C}}
\newcommand\SH{\mathscr{H}}
\newcommand\SB{\mathscr{B}}
\newcommand\SP{\mathscr{P}}
\newcommand\POWSET{\SP}%power set symbol
\newcommand\BF{\textbf}
\newcommand\IT{\textit}
\newcommand\UL{\underline}
\newcommand\DUL{\underline\underline}
\newcommand\OL[1]{$\overline{\mbox{#1}}$}
\newcommand\ITBF[1]{\textbf{\textit{#1}}}
\newcommand\ITUL[1]{\underline{\textit{#1}}}
\newcommand\BFUL[1]{\textbf{\underline{#1}}}
\newcommand\ITBFUL[1]{\textbf{\textit{\underline{#1}}}}
\newcommand\VEC{\bm}
\newcommand\GIVEN{\mid}%for conditional probabilities
\newcommand\UNI{u}%symbol for uniform random smapeling
\newcommand\DEF{:=}

\newcommand\bigo[1]{\mathcal{O}(#1)}

\newcommand\comment[1]{\footnote{\color{red}#1}}
\newcommand\npara{
\setlength{\parindent}{0ex}
\setlength{\parskip}{0.5em}
}
\newcommand\thmpara{
\setlength{\parindent}{0ex}
\setlength{\parskip}{0em}
}

\newcommand\BI[5]{
\bibitem[#1]{#1}
    #2.
    \newblock {\em #3}.
    \newblock #4.
    \newblock {\em #5}.
}

\pagestyle{fancyplain}
\chead{}
\lhead{}
\rhead{}
\rfoot{}

\newtheorem{theorem}{Theorem}
\newtheorem{lemma}[theorem]{Lemma}
\newtheorem{corollary}[theorem]{Corollary}
\newtheorem{definition}[theorem]{Definition}
\newtheorem{claim}[theorem]{Claim}
\newtheorem{fact}[theorem]{Fact}
\newtheorem{remark}[theorem]{Remark}
\newtheorem{problem}[theorem]{Problem}

\allowdisplaybreaks


%\begin{document}

\subsection{Shape \((3,2)\)}
Hilbert basis of \(\SF\SL_{(3,2)}^T\) is 
    \input{examples/shape32hbex}
\begin{lemma}
    \(x_1+x_2,x_1x_2 \in \IC[t_0,t_1,t_2,t_3,t_4]\) using straightaining relations. Hence \(z = x_1-x_2\) is integral over \(\IC[t_0,t_1,t_2,t_3,t_4,z]\) and Krull dimension of \(\IC[t_0,t_1,t_2,t_3,t_4,z]\) is 5.
\end{lemma}
\begin{proof}
    We have following two straightainning relations (see apendix for details):
    \[x_1x_2  = f_1(t_0,t_1,t_2,t_3,t_4,)\]
    \[x_1+x_2 = f_2(t_0,t_1,t_2,t_3,t_4,)\]
    this implies \(x_1-x_2\) is integral over \(\IC[t_0,t_1,t_2,t_3,t_4]\).
\end{proof}

\section{Relations: Flag varieties in \(n = 5\)}
\subsection{Shape \((r,1^s)\)}
In this subsection we will assume that \(\lambda = (r,1^s)\) and \(r+s=n\). 
%We have following lemma
\eat{
\begin{lemma}
    Let \(S\in STab(\lambda,n,\SC)\) of degree \(d\) 
    where \(\SC\) is maximal chain in Bruhat poset \((I(\lambda,n), \le)\).
    For any column \(\UL{p}\) of length \(r\) in \(S\), 
    there is factor of \(S\) of degree \(1\) which contain \(\UL{p}\).
\end{lemma}
\begin{proof}
    Shape and weight of \(S\) are \((r^d,1^{sd})\) and \(d\VEC{1}\) respectively.
    Let \(S = X_1X_2\) such that shape of \(X_1\) is \((r^d)\) and shape of \(X_2\) is \((1^{sd})\).
    Let \(\UL{p}\) be a column in \(X_1\) and \(i \in [n]\) such that 
    \(i\) do not appear in \(\UL{p}\). We have that weight of \(i\) in \(X_1\) is at most \(d-1\)
    (There are exactly \(d\) columns in \(X_1\) and entries in colums are distinct). 
    This implies weight of \(i\) in \(X_2\) is at least \(1\) 
    (weight of \(i\) in \(S=X_1X_2\) is \(d\)).

    Factor of \(S\) of degree \(1\) is constructed by taking any column \(p\) in \(X_1\) 
    and all \(i'\)s not appearing in \(\UL{p}\) from \(X_2\).
\end{proof}
\begin{corollary}
    Flag variety \(\SF\SL_{\lambda}\) is projectively normal.\qed
\end{corollary}
}
We define linear map \(\phi_d:\IC[\SF\SL_{\lambda}]_d^T \to \IC[\grass{n-1}{r-1}]_d\) as follows
\begin{align*}
    S=
    \begin{array}[c]{*{6}c}\cline{1-6}
        \lr{S_{11}}&\lr{\cdots}&\lr{S_{1d}}&\lr{S_{1(d+1)}}&\lr{\cdots}&\lr{S_{1(d+sd)}}\\\cline{1-6}
        \lr{S_{21}}&\lr{\cdots}&\lr{S_{2d}}\\\cline{1-3}
        \lr{\vdots}&\lr{\vdots}&\lr{\vdots}\\\cline{1-3}
        \lr{S_{r1}}&\lr{\cdots}&\lr{S_{rd}}\\\cline{1-3}
    \end{array}
    &\mapsto S'=
    \begin{array}[c]{*{3}c}\cline{1-3}
        \lr{S_{21}}&\lr{\cdots}&\lr{S_{2d}}\\\cline{1-3}
        \lr{\vdots}&\lr{\vdots}&\lr{\vdots}\\\cline{1-3}
        \lr{S_{r1}}&\lr{\cdots}&\lr{S_{rd}}\\\cline{1-3}
    \end{array}
\end{align*}
Tableau \(S'\) is obtained by removing first row in tableau \(S\). Weight of \(1\) in tableau \(S\) is \(d\) hence weight of \(1\) in \(S'\) is \(0\). We get \(\phi_d\) is isomorphism of vector spaces. 
\begin{observation}
    Let \(X,Y \in STab(\lambda,n,\SC)\) be two tableaux of degree \((d_1,d_2\) respectively.
    \(S=XY\) is nonstandard iff there are two columns of length \(r\) in \(S\) 
    which are noncomparable and both column contain \(1\) in first box.
\end{observation}
\begin{proof}
    Reverse direction is trivial. 
    To prove forward direction observe that there are \(d_1+d_2\) columns of length \(r\) in \(S\)
    and first entry in all of these columns is \(1\). 
    Hence each of the column of length \(r\) in \(S\) is comparable with any column of length \(1\). 
    Given that \(S\) is nonstandard and any two columns of length \(1\) are comparable 
    implies there must be two columns of length \(r\) which are not comparable.
\end{proof}

\begin{corollary}
    There is bijection between ideal generators of \(T\) quotient of flag variety \(\lambda=(r,1^s)\) and \(\grass{r-1}{n-1}\).
\end{corollary}
\begin{corollary}
    \(T\) quotient of flag variety \(\lambda=(r,1^s)\) is isomorphic to \(\grass{r-1}{n-1}\). In perticular this settles the question of \(T\) quotient shapes \((4,1), (3,1,1), (2,1,1,1)\) when \(n=5\).
\end{corollary}


%\end{document}
