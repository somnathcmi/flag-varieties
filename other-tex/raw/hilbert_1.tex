
\section{Hilbert Basis}

\label{sec:hilbert}
In this section we use ideas from polyhedral combinatorics to compute the generating set for 
$STab(\lambda,n,\SC)$.  We begin with some definitions. 
A standard and wonderful reference is Schrijver's book \cite{schrijver1998theory}. 
To compute the Hilbert basis we used the algorithm given in \cite{hemmecke2002computation}. 
We do not go into details of that algorithm, except mention how we used it.
Details of that algorithm for the specific problem of computing a generating set of $R = \oplus_m$ are described in a companion paper \cite{SS21}. 
We continue to use the notation
$\SC, A_{\SC},B_{\SC}$ from the previous section. To simolify notation we use $A$ instead of $A_{\SC}$.

\npara
\eat{
\begin{definition}[Lattice]
    A lattice in \(\IR^n\) is a discrete additive subgroup of \(\IR^n\). A lattice $L$  is said to be generated by $\{\ell{l}_1,\ldots,\ell{l}_d\}$ if every element in
    $L$ can be expressed as a linear combination of the $\ell{l}_i$s with coefficients in ${\mathbb Z}$ and if the generating set is linearly independent.
\end{definition}

Recall the definition of the Hermite Normal form (HNF) of a rational matrix, see \cite[Section 4.1]{schrijver1998theory} 
\begin{definition}[HNF]
A matrix of full row rank is said to be in Hermite normal form if it is of the form $[B\ \  0]$ where $B$ is a nonsingular, lower triangular, nonnegative matrix, in which each row 
has a unique largest element located on the main diagonal of $B$.
\end{definition}

We recall the following theorem, \cite[Theorem 4.1]{schrijver1998theory} 
\begin{theorem}
Every rational matrix of full row rank can be brought into HNF by a sequence of elementary column operations. There is a unimodular matrix $U$ such that $AU = [B\ \ 0]$.
\end{theorem}

We apply the above theorem to the matrix $A$ which is full row rank. So we get $AU = [B \ \ 0]$.We know from Section~\ref{sec:formulation}  that it is lattice points in the kernel of the matrix 
$A$ that are of interest to us. The column span of the last $r(n-r) - n$ columns of $U$ are clearly in the kernel of $A$. They are linearly independent and span the kernel of $A$. 

We recall a few more definitions which we will need in order to describe how to compute a generating set for $STab(r,n,\SC)$.
}
\begin{definition}[Cone]
    A set \(P\in \IR^n\) is cone if for any \(\VEC{x},\VEC{y}\in P\) 
    and non-negative \(\lambda_1,\lambda_2\),
    we have \(\lambda_1 \VEC{x} + \lambda_2 \VEC{y} \in P\). 
    A cone \(P\) is pointed if \(P \cap -P = \{0\}\). 
    A cone \(P\) is polyhedral if 
    \(P = \{ \VEC{x}\in \IR^n | B\VEC{x} \ge 0\}\) for some matrix \(B\).     
    A set $\{\VEC{g}_1,...\VEC{g}_k\} \subset P$ is called a (conical) generating set  of \(P\) 
    if for all \(\VEC{x} \in P\) there are non-negative \(\lambda_1,...,\lambda_k\) 
    such that \(\VEC{x} = \sum_{i=1}^k \lambda_i \VEC{g}_i\).
\end{definition}

\begin{definition}[Hilbert basis]
    Let \(P \in \IR^n\) be a polyhedral cone with rational generators. \eat{and 
    let \(\Lambda \subset \IZ^n\) be a lattice.}
    We call a finite set \(H = \{\VEC{h}_1,...,\VEC{h}_t\} \subset \IZ^n \cap P\) 
    a Hilbert basis of $P$ if for every integral vector $v \in P$ 
    there are non-negative integers \(\lambda_1,...,\lambda_t\) such that 
    $v=\sum_{i=1}^t \lambda_i h_i.$ 
   \end{definition} 
   
Every rational polyhedral cone $P$ has a Hilbert basis and if $P$  is pointed cone 
then there is unique inclusion minimal Hilbert basis \cite[Theorem 16.4]{schrijver1998theory}.
The following corollary is immediate.

\begin{corollary}
    $P_{\SC} = ker(A_{\SC}) \cap ker(B_{\SC}) \cap {\mathbb R}^{r(n-r)}_+$ is a pointed cone. 
    So $P_{\SC}$ has a unique inclusion minimal Hilbert basis.
\end{corollary}
 
\eat{Addition of two non-negative vectors in \(ker(A)\) is non-negative and in \(ker(A)\). 
2. Non-negative scalar multiple of non-negative vector in \(ker(A)\) 
is non-negative and in \(ker(A)\). 
3. \(x,-x \in P_{\SC} \implies x = 0\). }

\begin{definition}[Direct spliting]
    We say \(T \in STab(\lambda,n,\SC)\) splits directly 
    if there are two tableaux \(X,Y \in STab(\lambda,n,\SC)\) such that \(T=XY\) up to rearranging columns.
\end{definition}

For a maximal chain $\SC$, let $H_{\SC}$ denote the unique Hilbert basis. The following lemma is now immediate.
\begin{lemma}
    \(T \in STab(\lambda,n,\SC)\) splits directly iff \(\VEC{v}_{_T} \not\in H_{\SC}\). 
\end{lemma}
\begin{proof}
    \((\implies)\)
    Let \(T\) splits directly then there exist \(X,Y \in STab(\lambda,n,\SC)\) such that 
    \(T = XY\) up to rearranging columns. 
    This implies \(\VEC{v}_{_T} = \VEC{v}_{_X} + \VEC{v}_{_Y}\) 
    and thus \(\VEC{v}_{T}\) is decomposible hence \(\VEC{v}_{_T} \not\in H\).

    \((\impliedby)\) If \(\VEC{v}_{_T} \not\in H\) then there are non-negative integers 
    \(\lambda_1,...,\lambda_t\) such that \(\VEC{v}_{_T} = \sum_{i=1}^t \lambda_i \VEC{h}_i\). 
    Note that \(\sum_{i=1}^t \lambda_i \ge 2\) otherwise \(\VEC{v}_{_T} \in H\). 
    This implies there is nonzero integer \(\VEC{u} \in P_{\SC} \) and \(\VEC{h} \in H\) 
    such that \(\VEC{v}_{_T} = \VEC{h} + \VEC{u}\). 
    Hence we get \(T=T_{\VEC{h}}T_{\VEC{u}}\).     
\end{proof}
Here onwards, by {\em Hilbert basis of variety} we mean \(\cup_{\SC} \SH_{\SC}\) where union is over all maximal chains in bruhat poset \(B\).

\begin{corollary}[Direct Factoring Algorithm]
    There is algorithm which takes \(T\)-invariant tableau \(S\) and produces tableaux \(X,Y\) 
    such that \(S=XY\) and following holds:
    \begin{enumerate}
        \item If \(S\) splits directly then \(X,Y\) are proper \(T\)-invariant factors of \(S\)
        \item If \(S\) do not splits directly then \(X=S\) and \(Y=1\).
    \end{enumerate}
\end{corollary}
\begin{proof}
    Let \(\SC\) be the chain of all columns in \(S\) and \(H_{\SC}\) be the Hilbert basis of cone \(P_{\SC}\) which can be computed using alhorithm given in []. If \(S\not\in H_{\SC}\) then tableau do not factor directly, In this case our algorithm return \(X=S\) and \(Y=1\). Otherwise we find hilbert basis elements which are factor of \(S\). 
\end{proof}

