\section{Preliminaries}
 
We recall some notions from standard monomial theory that we will use, see \cite{seshadriintroduction} and \cite{lakshmibai2007standard}. The main theorem of standard monomial theory for $Gr_{r,n}$ allows us to describe the subring of $T$ invariants of $Gr_{r,n}$ for the Pl\"{u}cker embedding and for flag varieties in general. We then recall a theorem proved independently by Skorobogatov~\cite{skorobogatov1993swinnerton} and Kannan~\cite{kannan1998torus}, alluded to in the background section, which gives sufficient conditions for when semistability and stability coincide for the action of $T$ on $Gr_{r,n}$.
  
\subsection{Standard monomials}
 \label{sec:std:monom}
 We define a partial order "$\leq$" on $I_{r,n}$ by defining $ {\bf \underline{i}} \leq {\bf \underline{j}} \Leftrightarrow i_t \leq j_t$ for all $t = 1,2,\cdots,r$. The set $I_{r,n}$ with this partial order is called the Br\"{u}hat poset.  It will be useful to refine the partial order to a total order which we will also denote by the same symbol. Then there is a natural lexicographic order on ${\mathbb C}[\bigwedge^r {\mathbb C}^n]$ which is a 
 monomial order.
 
Denote by $\{p_{\bf \underline{i}} | {\bf \underline{i}} \in I_{r,n}\}$ the dual basis of the canonical basis in  $(\bigwedge^r{\mathbb C}^n)^*$. The $p_{\bf \underline{i}}$'s are called Pl\"{u}cker coordinates on $Gr_{r,n}$. The homogeneous coordinate ring of ${\mathbb C}[Gr_{r,n}]$ is the quotient of ${\mathbb C}[p_{\bf \underline{i}} | {\bf \underline{i}} \in I_{r,n}]$ by the ideal of polynomials $I(Gr_{r,n})$ vanishing on $Gr_{r,n}$ for its embedding in ${\mathbb P}(\bigwedge^r{\mathbb C}^n)$.  
  
 A monomial $p_{{\bf \underline{i}}^1} p_{{\bf \underline{i}}^2} \cdots p_{{\bf \underline{i}}^s}$ of degree $s$ in the Pl\"{u}cker coordinates is said to be standard in ${\mathbb C}[\bigwedge^r {\mathbb C}^n]$ iff 
 ${\bf \underline{i}}^1 \leq {\bf \underline{i}}^2 \cdots \leq {\bf \underline{i}}^s$.  If a product of monomials $p_{\bf \underline{i}} p_{\bf \underline{j}}$ is not standard then it is known that 
 \[ p_{\bf \underline{i}} p_{\bf \underline{j}} = p_{\bf \underline{i} \cup \bf \underline{j}}p_{\bf \underline{i} \cap \bf \underline{j}} + \text{other quadratic terms\ \ } \text{modulo\ \ }  I(G_{r,n}). \]
 Here $({\bf \underline{i} \cup \bf \underline{j}})_t = max(i_t, j_t), t = 1,\ldots,r$ and  $({\bf \underline{i} \cap \bf \underline{j}})_t = min (i_t, j_t), t = 1,\ldots,r$. 
 Furthermore the other quadratic terms in the above expression are products of monomials which are strictly bigger than $p_{\bf \underline{i}} p_{\bf \underline{j}}$ in the monomial order on ${\mathbb C}[\bigwedge^r {\mathbb C}^n]$. Quadratic terms in the above expression which are nonstandard can be refined again using the same rule and, since the terms are increasing in lex order, we see that any non standard monomial has an expression in terms of standard monomials. The difference between the two expressions goes by the name "straightening law" and this is an element of the ideal $I(Gr_{r,n})$. As a result we have the following theorem, see \cite[Proposition 1.3.6]{seshadriintroduction}.
 
 \begin{theorem} 

\noindent
\begin{enumerate}
 \item Standard monomials of degree $m$, $m \geq 0$, form a basis of the homogeneous coordinate ring of the Grassmannian.
 \item The ideal $I(Gr_{r,n})$ is generated by straightening laws.
 \end{enumerate}
\end{theorem}

Let $w \in W^{P^{\alpha_r}}$ corresponds to the element ${\bf \underline{i}}  \in I_{r,n}$. The Schubert variety $X(w) \subset Gr_{r,n}$ is given by the vanishing of the $p_{\bf \underline{j}}, {\bf \underline{j}} \not \leq  {\bf \underline{i}}$. It follows from \cite[Proposition 1.4.5]{seshadriintroduction} that standard monomials $p_{{\bf \underline{i}}^1} p_{{\bf \underline{i}}^2} \cdots p_{{\bf \underline{i}}^s}$ of degree $s$ with ${\bf \underline{i}}^s \leq {\bf \underline{i}}$ span a basis of $H^0(X(w), s \omega_r)$.


\subsubsection{$T$-invariants}
\label{sec:hb_grass}

\label{sec:torus_action}
Let \(p_{{\UL i}_1}^{a_{{\UL i}_1}}\cdots p_{{\UL i}_s}^{a_{{\UL i}_s}}\) be a monomial in \(\IC[\bigwedge^r \IC^n]\) of degree \(s\). We associate with this monomial a column standard tableau \(S\) of shape \((\sum_j  a_{{\UL i}_j} )^r  \) to this monomial by stacking $a_{{\UL i}_1}$ copies of ${\UL i}_1$ then $a_{{\UL i}_2}$ copies of ${\UL i}_2$ and so on, from left to right. Conversely for any column standard rectangular tableau \(S\) of shape \((\sum_j  a_{{\UL i}_j} )^r  \)  we get a monomial $p_{S}$ of degree \(s\) in \(\IC[\bigwedge^r\IC^n]\) by taking the product of the Pl\"{u}cker coordinates 
indexed by each column.  

Denote by \(p_{S}\) the monomial associated to a column standard tableau \(S\). The weight of a tableau \(S\) is defined to be the
\(n\)-dimentional vector \(wt(S)\), such that any \(i \in [n]\) appears \(wt(S)_i\) times in \(S\). The weight of a monomial in \(\IC[\bigwedge^r\IC^n]\) is the weight of the column standard tableau corresponding to the monomial.  

It is easy to describe the action of \(T\) on \(\IC[\bigwedge^r\IC^n]\).  For any \(t = diag(t_1,\cdots,t_n) \in T\) and \({\UL i} \in I_{r,n}\) define \(t\cdot p_{\UL i} = (t_{i_1}\cdots t_{i_r})^{-1}p_{\UL i}\).  It follows that for any monomial \(p_{_S}\), \(t\cdot p_{_S} = (t_i^{-wt(S)_1}\cdots t_n^{-wt(S)_n})p_{_S}\). 
We have, 
\begin{proposition}
\label{prop:tinv}
 A monomial \(p_{_S} \in \IC[\bigwedge^r\IC^n]\) is \(T\)-invariant iff \(wt(S)\) is a uniform vector i.e. if \(wt(S)\)  is a vector of the form $(k,k,\ldots,k)$ for some positive integer $k $.
\end{proposition}

We recall the following proposition, see Skorobogatov~\cite{skorobogatov1993swinnerton} and Kannan~\cite{kannan1998torus}.
\begin{proposition}
Let $(r,n)=1$. Then we have $(G/P)^{ss}_{T}({\mathcal L}(\omega_r)) = (G/P)^{s}_{T}({\mathcal L}(\omega_r)).$
\end{proposition}

If all semistable points are stable, the GIT quotient would in fact be a geometric quotient, with each point in the quotient variety corresponding to an orbit under $T$. We will assume 
henceforth that $(r,n)=1$. In this setting if  \(p_{_S} = p_{{\UL i}_1} p_{{\UL i}_2} \cdots p_{{\UL i}_s}\) is $T$ invariant then 
\(s\) is multiple of \(n\) and \(wt(S) = (\frac{sr}{n},\cdots,\frac{sr}{n})\). It is easy to see that there are monomials of degree \(n\) in the Pl\"{u}cker coordinates which are \(T\)-invariant, see \cite{bakshi2020torus} for example. We have,

\begin{theorem}
Let $R(k) = H^0(G/P, {\cal L}(kn\omega_r))^T$. Then $\Qgmodp{{\cal L}(\omega_r)} = Proj(\oplus_k R(k))$.
\end{theorem}

This gives us a geometric description of the quotient. However we would like to say more, and for that we will need to compute an explicit set of generators of the ring $\oplus_k R(k)$
and relations between them. When $r=2$ this is known, see ~\cite{howard2005projective},~\cite{bakshi2020torus}. However this becomes difficult when $r$ is bigger than or equal to three since there are too many generators and this has to be done more systematically. This is what we address in the next section.

