\section{Hilbert basis}
\label{sec:hilbert}
In this section we use ideas from polyhedral combinatorics to compute a generating set for $STab(r,n,\SC)$.  We begin with some definitions. A standard and wonderful reference is Schrijver's book \cite{schrijver1998theory}. To compute the Hilbert basis we used the algorithm given in \cite{hemmecke2002computation}. We do not go into details of that algorithm, except mention how we used it.
Details of that algorithm for the specific problem of computing a generating set of $R = \oplus_m$ are described in a companion paper \cite{SS21}. We continue to use the notation
$\SC, A_{\SC}$ from the previous section. To simolify notation we use $A$ instead of $A{\SC}$.

\npara
\eat{\begin{definition}[Lattice]
    A lattice in \(\IR^n\) is a discrete additive subgroup of \(\IR^n\). A lattice $L$  is said to be generated by $\{\ell{l}_1,\ldots,\ell{l}_d\}$ if every element in
    $L$ can be expressed as a linear combination of the $\ell{l}_i$s with coefficients in ${\mathbb Z}$ and if the generating set is linearly independent.
\end{definition}

Recall the definition of the Hermite Normal form (HNF) of a rational matrix, see \cite[Section 4.1]{schrijver1998theory} 
\begin{definition}[HNF]
A matrix of full row rank is said to be in Hermite normal form if it is of the form $[B\ \  0]$ where $B$ is a nonsingular, lower triangular, nonnegative matrix, in which each row 
has a unique largest element located on the main diagonal of $B$.
\end{definition}

We recall the following theorem, \cite[Theorem 4.1]{schrijver1998theory} 
\begin{theorem}
Every rational matrix of full row rank can be brought into HNF by a sequence of elementary column operations. There is a unimodular matrix $U$ such that $AU = [B\ \ 0]$.
\end{theorem}

We apply the above theorem to the matrix $A$ which is full row rank. So we get $AU = [B \ \ 0]$.We know from Section~\ref{sec:formulation}  that it is lattice points in the kernel of the matrix 
$A$ that are of interest to us. The column span of the last $r(n-r) - n$ columns of $U$ are clearly in the kernel of $A$. They are linearly independent and span the kernel of $A$. 

We recall a few more definitions which we will need in order to describe how to compute a generating set for $STab(r,n,\SC)$.
}
\eat{   
 Every rational polyhedral cone $P$ has a Hilbert basis and if $P$  is pointed cone then there is unique inclusion minimal Hilbert basis \cite[Theorem 16.4]{schrijver1998theory}.}
 
 \eat{Addition of two non-negative vectors in \(ker(A)\) is non-negative and in \(ker(A)\). 
2. Non-negative scalar multiple of non-negative vector in \(ker(A)\) 
is non-negative and in \(ker(A)\). 
3. \(x,-x \in P_{\SC} \implies x = 0\). }

\begin{definition}
    We say \(T \in STab(r,n,\SC)\) splits directly 
    if there are two tableaux \(X,Y \in STab(r,n,\SC)\) such that \(T=XY\) up to rearranging columns.
\end{definition}

For a maximal chain $\SC$, let $H_{\SC}$ denote the unique Hilbert basis. The following lemma is now immediate.
\begin{lemma}
    \(T \in STab(r,n,\SC)\) splits directly iff \(v_{_T} \not\in H_{\SC}\). 
\end{lemma}
\begin{proof}
    \((\implies)\)
    Let \(T\) splits directly then there exist \(X,Y \in STab(r,n,\SC)\) such that 
    \(T = XY\) up to rearranging columns. This implies \(v_{_T} = v_{_X} + v_{_Y}\) 
    and thus \(v_{T}\) is decomposible hence \(v_{_T} \not\in H\).

    \((\impliedby)\) If \(v_{_T} \not\in H\) then there are non-negative integers 
    \(\lambda_1,...,\lambda_t\) such that \(v_{_T} = \sum_{i=1}^t \lambda_i h_i\). 
    Note that \(\sum_{i=1}^t \lambda_i \ge 2\) otherwise \(v_{_T} \in H\). 
    This implies there is \(u\in Z^{l+1}\cap P_{\SC} \) and \(h,h' \in H\) 
    such that \(v_{_T} = h + h' + u\). 
    Since \(v_{_{ST}} = v_{_S} + v_{_T}\) we get \(T=T^{h}T^{h'}T^u\).     
\end{proof}

In remaining document by the term "Hilbert basis of \(\grass{r}{n}\)" we mean \(\cup_{\SC} H_{\SC}\) where union is taken over all maximal chains in Bruhat poset. Similarly the term "Hilbert basis of schubert variety" is used.



%END

